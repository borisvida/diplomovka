%%%%%%%%%%%%%%%%%%%%%%%%%%%%%%%%%%%%%%%%%
% Beamer Presentation
% LaTeX Template
% Version 1.0 (10/11/12)
%
% This template has been downloaded from:
% http://www.LaTeXTemplates.com
%
% License:
% CC BY-NC-SA 3.0 (http://creativecommons.org/licenses/by-nc-sa/3.0/)
%
%%%%%%%%%%%%%%%%%%%%%%%%%%%%%%%%%%%%%%%%%

%----------------------------------------------------------------------------------------
%	PACKAGES AND THEMES
%----------------------------------------------------------------------------------------

\documentclass[slovak]{beamer}

\mode<presentation> {

% The Beamer class comes with a number of default slide themes
% which change the colors and layouts of slides. Below this is a list
% of all the themes, uncomment each in turn to see what they look like.

%\usetheme{default}
%\usetheme{AnnArbor}
%\usetheme{Antibes}
%\usetheme{Bergen}
%\usetheme{Berkeley}
%\usetheme{Berlin}
%\usetheme{Boadilla}
%\usetheme{CambridgeUS}
%\usetheme{Copenhagen}
%\usetheme{Darmstadt}
%\usetheme{Dresden}
%\usetheme{Frankfurt}
%\usetheme{Goettingen}
%\usetheme{Hannover}
%\usetheme{Ilmenau}
%\usetheme{JuanLesPins}
%\usetheme{Luebeck}
\usetheme{Madrid}
%\usetheme{Malmoe}
%\usetheme{Marburg}
%\usetheme{Montpellier}
%\usetheme{PaloAlto}
%\usetheme{Pittsburgh}
%\usetheme{Rochester}
%\usetheme{Singapore}
%\usetheme{Szeged}
%\usetheme{Warsaw}

% As well as themes, the Beamer class has a number of color themes
% for any slide theme. Uncomment each of these in turn to see how it
% changes the colors of your current slide theme.

%\usecolortheme{albatross}
%\usecolortheme{beaver}
%\usecolortheme{beetle}
%\usecolortheme{crane}
%\usecolortheme{dolphin}
%\usecolortheme{dove}
%\usecolortheme{fly}
%\usecolortheme{lily}
%\usecolortheme{orchid}
%\usecolortheme{rose}
%\usecolortheme{seagull}
%\usecolortheme{seahorse}
%\usecolortheme{whale}
%\usecolortheme{wolverine}

%\setbeamertemplate{footline} % To remove the footer line in all slides uncomment this line
%\setbeamertemplate{footline}[page number] % To replace the footer line in all slides with a simple slide count uncomment this line

%\setbeamertemplate{navigation symbols}{} % To remove the navigation symbols from the bottom of all slides uncomment this line
}

\input foja

\usepackage[utf8]{inputenc}
\usepackage{babel}
\usepackage{translator}
\usepackage{graphicx} % Allows including images
\usepackage{booktabs} % Allows the use of \toprule, \midrule and \bottomrule in tables
\usepackage{mathrsfs}
\usepackage{amsmath}
\usepackage{amssymb}

%----------------------------------------------------------------------------------------
%	TITLE PAGE
%----------------------------------------------------------------------------------------

\title[Transformácia s pomocnou informáciou]{Využitie transformácie pri riešení problémov s pomocnou informáciou}

\author{Boris Vida} 
\institute[FMFI UK]
{
Katedra informatiky, FMFI UK \\
\medskip
\textit{Prof. RNDr. Branislav Rovan, PhD.}
}
\date{10. februára 2015}

\begin{document}

\begin{frame}
\titlepage % Print the title page as the first slide
\end{frame}

\begin{frame}
\frametitle{Obsah}
\tableofcontents
\end{frame}

%----------------------------------------------------------------------------------------
%	PRESENTATION SLIDES
%----------------------------------------------------------------------------------------
\section{Motivácia a ciele}

\begin{frame}
\frametitle{Motivácia a ciele}

\begin{itemize}
\item prídavná informácia
\item zjednodušovanie výpočtov (orákulum, paralelizácia)
\item čo ak je informácia v nepoužiteľnom formáte?
\item zložitosť a-prekladačov
\item triedy rozložiteľných a nerozložiteľných jazykov a ich vlastnosti
\item porovnanie s predošlými výsledkami (bez možnosti transformácie)
\end{itemize}

\end{frame}

%----------------------------------------- end of section -----------------------------------------

\section{Definícia}

\begin{frame}
\frametitle{Popis koncepcie}

\begin{itemize}
\item rozhodujeme, či vstup $w$ patrí do (regulárneho) jazyka $L_{dec}$
\item rada: a-prekladač $M$ a (regulárny) jazyk $L_{adv}$
\item sľub, že $M(w) \subseteq L_{adv}$
\item predošlé výsledky bez $M$ - priamo informácia, že vstup patrí do $L_{adv}$
\end{itemize}

\begin{definition}
Pre a-prekladač $M$ a jazyk $L$, $M^{-1}(L)$ je množina všetkých slov takých, že ich obrazy pomocou $M$ sú podmnožinou $L$.
\end{definition}

\begin{itemize}
\item $M^{-1}$ sa dá simulovať duálnym a-prekladačom, navyše prídavné podmienky na vstupy mimo $L_{adv}$
\end{itemize}

\end{frame}

\begin{frame}
\frametitle{Popis koncepcie}

\begin{definition}
Pre jazyk $L_1$ a DKA $A = (K, \Sigma, \delta, q_0, F)$, $L[L_1](A) = L(A) \cap L_1$.
\end{definition}

\begin{definition}
Dvojica $(L_{adv},M)$ je \emph{efektívna T-rada pre $L_{dec}$}, ak existuje DKA $A_{simple}$ taký, že $L_{dec} = L[M^{-1}(L_{adv})](A_{simple})$ a $\mathcal{C}_{state}(A_{simple}) + \mathcal{C}_{state}(M) + \mathcal{C}_{state}(L_{adv}) \leq \mathcal{C}_{state}(L_{dec})$.
\end{definition}

\end{frame}

%----------------------------------------------
\section{Dosiahnuté výsledky}

\begin{frame}
\frametitle{Odhady počtu stavov}

Transformácie jazykov tvaru $L_{k} = \{ a^n| n = 0 (mod k)\} $
\begin{itemize}
\item[]
\item hľadáme a-prekladač $M$ taký, že $M(L_{k}) = L_{l}$
\item[]
\item[]
\item ak $k$ a $l$ sú nesúdeliteľné, $M$ má najmenej $l$ stavov
\item[]
\item ak $k \leq l$, potom $M$ má najmenej $\frac{l}{gcd(k,l)}$ stavov 
\item[]
\item ak $k > l$, potom $M$ má najmenej $min(l, \frac{k}{gcd(k,l)})$ stavov
\item[]
\item[]
\item zhrnutie: $\mathcal{C}_{state}(L_k, L_l) = \min (l, \frac{\max (k,l)}{\gcd (k,l)})$
\end{itemize}

\end{frame}

\begin{frame}
\frametitle{Porovnanie rozložiteľných a T-rozložiteľných jazykov}

\begin{itemize}
\item porovnanie s koncepciou bez a-prekladača - vznikajú iné triedy (unárna abeceda) - jednoprvkové jazyky
\item[]
\item T-rozložiteľné - napr. $\{ a^{12k} | k \geq 0 \}$
\item[]
\item T-nerozložiteľné - napr. $\{ a^{7k} | k \geq 0 \}, a^*$
\end{itemize}

\end{frame}

\begin{frame}
\frametitle{Uzáverové vlastnosti}
\begin{itemize}
\item uzáverové vlastnosti
\item deterministické operácie
\end{itemize}
\begin{table}
\begin{tabular}{l | c | c | c | c | c | c | c}
Trieda & $^c$ & $h_{\epsilon}$ & $h^{-1}$ & $\cup$ & $\cap$ & $+,*$ & $m+,m*$ \\
\hline
T-rozložiteľné & ? & n & n & n & n & n & a\\ 
T-nerozložiteľné & ? & n & ? & n & n & ? & ?
\end{tabular}
\caption{Uzáverové vlastnosti}
\end{table}

\end{frame}

%------------------------------------------------
\section{Možné smery ďalšieho výskumu}
\begin{frame}
\frametitle{Možné smery ďalšieho výskumu}

\begin{itemize}
\item triedy problémov s rovnakou pomocnou informáciou/rovnakým prekladom
\item[]
\item nutné/postačujúce podmienky na T-rozložiteľnosť
\item[]
\item zložitosť a-prekladačov
\end{itemize}
\end{frame}

%------------------------------------------------

\begin{frame}
\Huge{\centerline{Ďakujem za pozornosť}}
\end{frame}

\end{document} 
