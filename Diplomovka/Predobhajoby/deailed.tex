%%%%%%%%%%%%%%%%%%%%%%%%%%%%%%%%%%%%%%%%%
% Beamer Presentation
% LaTeX Template
% Version 1.0 (10/11/12)
%
% This template has been downloaded from:
% http://www.LaTeXTemplates.com
%
% License:
% CC BY-NC-SA 3.0 (http://creativecommons.org/licenses/by-nc-sa/3.0/)
%
%%%%%%%%%%%%%%%%%%%%%%%%%%%%%%%%%%%%%%%%%

%----------------------------------------------------------------------------------------
%	PACKAGES AND THEMES
%----------------------------------------------------------------------------------------

\documentclass[slovak]{beamer}

\mode<presentation> {

% The Beamer class comes with a number of default slide themes
% which change the colors and layouts of slides. Below this is a list
% of all the themes, uncomment each in turn to see what they look like.

%\usetheme{default}
%\usetheme{AnnArbor}
%\usetheme{Antibes}
%\usetheme{Bergen}
%\usetheme{Berkeley}
%\usetheme{Berlin}
%\usetheme{Boadilla}
%\usetheme{CambridgeUS}
%\usetheme{Copenhagen}
%\usetheme{Darmstadt}
%\usetheme{Dresden}
%\usetheme{Frankfurt}
%\usetheme{Goettingen}
%\usetheme{Hannover}
%\usetheme{Ilmenau}
%\usetheme{JuanLesPins}
%\usetheme{Luebeck}
\usetheme{Madrid}
%\usetheme{Malmoe}
%\usetheme{Marburg}
%\usetheme{Montpellier}
%\usetheme{PaloAlto}
%\usetheme{Pittsburgh}
%\usetheme{Rochester}
%\usetheme{Singapore}
%\usetheme{Szeged}
%\usetheme{Warsaw}

% As well as themes, the Beamer class has a number of color themes
% for any slide theme. Uncomment each of these in turn to see how it
% changes the colors of your current slide theme.

%\usecolortheme{albatross}
%\usecolortheme{beaver}
%\usecolortheme{beetle}
%\usecolortheme{crane}
%\usecolortheme{dolphin}
%\usecolortheme{dove}
%\usecolortheme{fly}
%\usecolortheme{lily}
%\usecolortheme{orchid}
%\usecolortheme{rose}
%\usecolortheme{seagull}
%\usecolortheme{seahorse}
%\usecolortheme{whale}
%\usecolortheme{wolverine}

%\setbeamertemplate{footline} % To remove the footer line in all slides uncomment this line
%\setbeamertemplate{footline}[page number] % To replace the footer line in all slides with a simple slide count uncomment this line

%\setbeamertemplate{navigation symbols}{} % To remove the navigation symbols from the bottom of all slides uncomment this line
}

\input foja

\usepackage[utf8]{inputenc}
\usepackage{babel}
\usepackage{translator}
\usepackage{graphicx} % Allows including images
\usepackage{booktabs} % Allows the use of \toprule, \midrule and \bottomrule in tables
\usepackage{comment}

%----------------------------------------------------------------------------------------
%	TITLE PAGE
%----------------------------------------------------------------------------------------

\title[Mnohohlavové automaty]{Mnohohlavové automaty}

\author{Boris Vida} 
\institute[FMFI UK]
{
Katedra informatiky, FMFI UK \\
\medskip
\textit{Prof. RNDr. Pavol Ďuriš, CsC.}
}
\date{\today}

\begin{document}

\begin{frame}
\titlepage % Print the title page as the first slide
\end{frame}

\begin{frame}
\frametitle{Obsah}
\tableofcontents
\end{frame}

%----------------------------------------------------------------------------------------
%	PRESENTATION SLIDES
%----------------------------------------------------------------------------------------
\section{Motivácia a ciele}

\begin{frame}
\frametitle{Motivácia}

\begin{itemize}
\item zložitosť formálnych jazykov - jedna z centrálnych oblastí teoretickej informatiky
\item narastajúci záujem o paralelné modely
\item mnohohlavové konečné automaty - prirodzené rozšírenie konečných automatov
\item súvis so známymi triedami - iný pohľad na otvorené problémy
\item rôzne obmedzenia - slepé hlavy, bezstavové automaty...
\item dátovo nezávislé automaty
\end{itemize}

\end{frame}

\begin{frame}
\frametitle{Ciele}

\begin{itemize}
\item vytvoriť prehľad známych výsledkov v oblasti mnohohlavových konečných automatov
\item pre niektoré z týchto výsledkov vypracovať alternatívny dôkaz
\item porovnať výpočtovú silu niektorých podtried
\item zameranie na dátovo nezávislé automaty
\end{itemize}

\end{frame}

%----------------------------------------- end of section -----------------------------------------

\section{Mnohohlavové automaty}

\subsection{Definícia}

\begin{frame}
\frametitle{Definícia - mnohohlavový konečný automat \cite{hkm}}

\begin{definition}
\emph{Mnohohlavový konečný automat} je šestica $A = (Q, \Sigma \cup \{ \vertcent , \$ \} , k, \delta , s_{0}, F)$, kde 
\begin{enumerate}
\item $Q$ je konečná stavová množina,
\item $\Sigma $\ je vstupná abeceda; $\vertcent , \$ \notin \Sigma $ je ľavá, resp. pravá zarážka 
\item $k$ je počet hláv,
\item $\delta$ je čiastočná prechodová funkcia $\delta : Q \times (\Sigma \cup \{ \vertcent , \$ \} )^k\to 2^{Q \times \{ -1, 0, 1\} ^k}$
\item $s_{0} \in \Sigma $ je počiatočný stav,
\item $F \subset Q$ is množina akceptačných stavov.
\end{enumerate}
\end{definition}

\begin{itemize}
\item počet hláv
\item determinizmus vs. nedeterminizmus
\item 1-smerné vs. 2-smerné
\end{itemize}

\end{frame}

% ---------------------------------------------- end of subsection ---------------------------------------------------
\subsection{Výpočtová sila}

\begin{frame}
\frametitle{Mnohohlavové automaty a logaritmický priestor}

\cveta
$\textbf{L} = \bigcup_{k \in \N } \mathcal{L} (2DFA(k)); \textbf{NL} = \bigcup_{k \in \N } \mathcal{L} (2NFA(k)) $ \\
\emph{Dôkaz.} \\
\begin{enumerate}
\item $\textbf{L} \supseteq \bigcup_{k \in \N } \mathcal{L} (2DFA(k)) $ \\
\begin{itemize}
\item použijeme viacpáskový automat
\item polohy hláv máme zapísané na jednotlivých páskach
\item Turingov stroj si postupne do stavu zozbiera čítané symboly
\item zmení stav a inkrementuje/dekrementuje príslušné pásky
\item akceptuje rovnako ako mnohohlavový automat
\end{itemize}
\end{enumerate}

\end{frame}

\begin{frame}
\frametitle{Mnohohlavové automaty a logaritmický priestor}

\begin{enumerate}
\setcounter{enumi}{1}
\item $\textbf{L} \subseteq \bigcup_{k \in \N } \mathcal{L} (2DFA(k)) $ \\
\begin{itemize}
\item myšlienka podobná simulácii Turingovho stroja na dvoch zásobníkoch
\item max. číslo zapísané na páske je $m^{n}$ (m je veľkosť pracovnej abecedy TS)
\item hlavy používame ako koeficienty polynómu $\sum_{j=0}^{k_1} r_{j}*m^{j}$, ktorý uchováva hodnotu "čísla" zapísaného vľavo, resp. vpravo od hlavy na pracovnej páske TS ($r_{j}$ označuje polohu $j$-tej hlavy, ${k_1}$ je približne polovica počtu hláv)
\item symbol pod hlavou TS sa uchováva v stave
\item čítanie zodpovedá deleniu so zvyškom, prepis pripočítavaniu/odpočítavaniu
\end{itemize}
\end{enumerate}

Otvorený problém $\textbf{L}$ vs $\textbf{NL}$.
\end{frame}


% ---------------------------------------------- end of subsection ---------------------------------------------------

\subsection{Poddtriedy}

\begin{frame}
\frametitle{Podtriedy mnohohlavových konečných automatov}

\begin{itemize}
\item deterministické a nedeterministické (ostrá inklúzia pre jednosmerné, pre dvojsmerné otvorený problém)
\item jedno- a dvojsmerné (zrkadlový jazyk)
\item podľa počtu hláv (ostré inklúzie, pre dvojsmerné automaty svedčia unárne jazyky) \cite{yr}
\item dátovo nezávislé
\item uniforminé a neuniformné
\item ďalšie modely - slepé, zametajúce, bezstavové automaty...
\end{itemize}
\end{frame}

%=============================================== end of section ====================================================

\section{Dátovo nezávislé mnohohlavové automaty}

\subsection{Definícia}

\begin{frame}
\frametitle{Definícia a základné vlastnosti - dátovo nezávislé}

\begin{definition}
Mnohohlavový automat nazývame \emph{dátovo nezávislý}, ak pozícia $i$-tej čítacej hlavy po $k$ krokoch výpočtu na vstupnom slove $w$ závisí len od $i$, $\left| w \right|$ a $k$. 
\end{definition}
\begin{itemize}
\item ekvivalentné s $NC^{1}$ \cite{mh}
\item patria sem všetky unárne jazyky z L
\item nedeterminizmus nepomáha
\item jedno- a dvojsmerné (zrkadlový jazyk)
\item počet hláv (pre dvojsmerné svedčia unárne jazyky, pre jednosmerné otvorený problém)
\item najlepší výsledok - $\mathcal{L} (1DiDFA(\sqrt{2}.k)) \subsetneq \mathcal{L} (1DiDFA(2k+2))$ pre $k \geq 1$ \cite{dur}
\end{itemize}
\end{frame}

%------------------------------------------------

\subsection{Otázky zložitosti}
\begin{comment}
\begin{frame}
\frametitle{Otázky zložitosti}
\cveta
$\mathcal{L} (1DFA(2)) - \bigcup_{k \in \N } \mathcal{L} (1DiDFA(k)) \neq \emptyset$ \\
\emph{Dôkaz.} \\*
\begin{itemize}
\item technika dôkazu z \cite{yr}
\item použijeme jazyk $L = \{ w\#^{+}w\#^{+} | w \in \{ a, b \}^{*}, k \geq 1 \}$
\item zrejme patrí do $\mathcal{L} (1DFA(2))$
\item nech existuje $1DiDFA(k)$, ktorý akceptuje $L$
\item rozdelíme vstup na bloky - musia byť snímané paralelne
\item ukážeme, že pri dostatočne dlhých vstupoch existuje dvojica blokov, ktoré ani raz počas výpočtu nie sú snímané paralelne
\end{itemize}
\end{frame}

%------------------------------------------------

\begin{frame}
\frametitle{Otázky zložitosti}
\begin{itemize}
\item využijeme podmnožinu $L_{h} = \{ w \# ^{i.p} w \# ^{j.p} | w \in \{ a, b \} ^{h.p} \wedge i + j = 2.h \}$, kde $h=4.k^{2}+1$ a $p$ je konštanta dostatočne veľká vzhľadom na počet stavov automatu
\item teda vstup pozostáva z  $4.h$ blokov
\item pre blok $u_{s}$ máme $i+j = 2.h$ možností, kde sa môže nachádzať korešpondujúci blok $u_{s'}$
\item pre $h$ rôznych hodnôt $s$ je to teda $h.2.h$ možností
\item jedna dvojica hláv vie pokryť iba $8.h - 1$ dvojíc $(u_{s}, u_{s'})$
\item $\binom{k}{2}$ dvojíc hláv pokráva menej než $k^{2}.8.h$ dvojíc
\item podľa voľby $h$ teda naozaj nepokrytá dvojica blokov existuje, označme $(x_{t}, x_{t'})$
\end{itemize}
\end{frame}

\begin{frame}
\frametitle{Otázky zložitosti}
\begin{itemize}
\item slová z $L_{h}$ rozdelíme do tried podľa stavu, v ktorom sa automat nachádza pri začatí čítania bloku $x_{t'}$
\item jazyk je "hustý", veľa slov v jednej triede, vieme zmiešať dva vstupy na taký, ktorý nepatrí do jazyka
\end{itemize}
Pre dvojsmerné automaty by vyriešenie tejto otázky dalo odpoveď na vzťah tried $\textbf{NC} ^{1}$ a $\textbf{L}$ (resp. $\textbf{NL}$).
\end{frame}
\end{comment}

\begin{frame}
\frametitle{Otázky zložitosti}
\cveta
$\mathcal{L} (1DFA(2)) - \bigcup_{k \in \N } \mathcal{L} (1DiDFA(k)) \neq \emptyset$ \\
\emph{Dôkaz.} \\*
\begin{itemize}
\item použijeme jazyk $L = \{ w\#^{+}w\#^{+} | w \in \{ a, b \}^{*}, k \geq 1 \}$
\item zrejme patrí do $\mathcal{L} (1DFA(2))$
\item nech existuje $1DiDFA(k)$, ktorý akceptuje $L$
\item rozdelíme vstup na bloky - musia byť snímané paralelne
\item pri dostatočne dlhých vstupoch existuje dvojica blokov, ktoré ani raz počas výpočtu nie sú snímané paralelne
\item vieme skombinovať dva vstupy na taký, ktorý nepatrí do jazyka a je akceptovaný našim automatom - spor
\end{itemize}
\end{frame}

\begin{frame}
\frametitle{Otázky zložitosti}
Označme konečný automat s jednou obyčajnou a jednou slepou hlavou ako $\mathcal{L} (1DPBFA(2))$.
\cveta
$\mathcal{L} (1DPBFA(2)) - \mathcal{L} (1DiDFA(2)) \neq \es$.
\end{frame}

%------------------------------------------------
\section*{Literatúra}

\begin{frame}
\frametitle{Literatúra}
\footnotesize{
\begin{thebibliography}{99}
\bibitem[1]{hkm} Markus Holzer, Martin Kutrib, Andreas Malcher (2010)
\newblock Complexity of multi-head finite automata: Origins and directions
\newblock \emph{Theoretical Computer Science} 412(2011), 83-96.

\bibitem[2]{mh} Markus Holzer (2002)
\newblock Multi-head finite automata: data-independent versus data-dependent computations
\newblock \emph{Theoretical Computer Science} 286(2002), 97-116.

\bibitem[3]{yr} Andrew C. Yao, Ronald L. Rivest (1978)
\newblock k+1 heads are better than k
\newblock \emph{Journal of the ACM} 25(2), 337-340.

\bibitem[4]{dur} P. Ďuriš. (2012)
\newblock A note on the hierarchy of one-way data-independent multi-head finite automata
\newblock In \emph{Proceedings of Electronic Colloquium on Computational Complexity} (ECCC), (2012) 92-92
\end{thebibliography}
}
\end{frame}

%------------------------------------------------

\begin{frame}
\Huge{\centerline{Ďakujem za pozornosť}}
\end{frame}

%----------------------------------------------------------------------------------------

\end{document} 
