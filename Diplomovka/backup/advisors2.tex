\chapter{Foreign advisors}
\label{chap:advisors}

\section{Description of the framework}
\paragraph{}
We now proceed to definitions associated to the central matter of our thesis, which is the framework for using transformation in provelm solving with advisory information. We provide three slightly different definitions, which will later be compared in their properties. They differ in the used transformation concept. The first two frameworks use a general and widely used model of (nondeterministic) a-transducer, while the third one relies on (deterministic) sequential transducers.

\paragraph{}
\cdefinicia Let $M$ be an a-transducer and $L$ a language. Then $M_{\forall}^{-1}(L)$ is the set of all words, such that all their images belong to $L$. Formally \\
\centerline{$M_{\forall}^{-1}(L) = \{ w | M(w) \neq \emptyset \wedge M(w) \subseteq L \}$.}

\paragraph{}
\cdefinicia Let $M$ be an a-transducer and $L$ a language. Then $M_{\exists}^{-1}(L)$ is the set of words $w$, such that there is at least one image of $w$ belonging to $L$. Formally \\
\centerline{$M_{\exists}^{-1}(L) = \{ w | M(w) \cap L \neq \emptyset\}$.}


\paragraph{}
\cpriklad Let $M = (\{q_0,q_1\}, \{a\},\{b\},H,q_0,\{q_0,q_1\})$ be an a-transducer, where $H = \{(q_0,a,b,q_1),(q_1,\varepsilon,b,q_0)\}$. Moreover, let $L = \{b^2\}^*$. Every word from $a^k \in \{a\}^+$ has two images: $M(a^k) = \{b^{2k-1},b^{2k}\}$ and $M(\varepsilon) = \{\varepsilon\}$. Therefore, $M_{\forall}^{-1}(L) = \{a\}^*$,  while $M_{\exists}^{-1}(L) = \{\varepsilon\}$. 

\paragraph{}
\cdefinicia Let $L_{dec}$ be a regular language. Let $q \in \{\forall,\exists\}$. A pair $(L_{adv}, M)$, where $L_{adv}$ is a regular language and $M$ an a-transducer is called an \emph{$NT_{q}$-advice with regard to $L_{dec}$}, if there exists a deterministic finite automaton $A'$, such that $L_{dec} = L[M_{q}^{-1}(L_{adv})](A')$. Moreover, $(L_{adv}, M)$ is called \emph{effective}, if $\C{A'} + \C{M} + \C{L_{adv}} \leq	 \C{L_{dec}}$.

\paragraph{}
\cpriklad Let $L_{dec} = \{ a^{12k}| k \geq 0 \} $. Let $M= (\{q_0, q_1\}, \{a\}, \{a\}, H, q_0, \{q_0\})$, where $H = \{(q_0, a, a, q_1), (q_1, a, \varepsilon, q_0)\}$ and $L_{adv} = \{ a^{2k}| k \geq 0 \}$. $M$ shortens every word from $a^*$ to half of its length, so it is easy to see, that $M_{\forall}^{-1}(L_{adv}) = M_{\exists}^{-1}(L_{adv}) = \{ a^{4k}| k \geq 0 \}$. We now construct a simpler finite automaton $A'$ for the language $L_{simple} = \{ a^{3k}| k \geq 0 \}$. Clearly, $\C{A'} + \C{M} + \C{L_{adv}} = 3 + 2 + 2 \leq 12 = \C{L_{dec}}$ and $L[M^{-1}_{\forall}L_{adv}](A') = L[M^{-1}_{\exists}L_{adv}](A') = L_{dec}$ which means, that $L_{adv}$ with $M$ is an effective $NT_{\forall}$- and $NT_{\exists}$-advice with regard  to $L_{dec}$.

\paragraph{}
\cpriklad Let $L_{dec} = \{ a^{3k}|k \geq 0 \} \cup \{ a^{5k}|k \geq 0\}$. Now, let $M$ be an a-transducer from Figure 4.1, $L_{adv} = L_{simple} = \{a\}^*$. We can see, that $M$ has accepting computations only on words from $L_{dec}$ and so $M^{-1}_{\exists}(L_{adv}) = M^{-1}_{\forall}(L_{adv}) = L_{dec}$. $\C{M} + \C{L_{simple}} + \C{L_{adv}} = 11$, while $\C{L_{dec}} = 15$ (this could be proven by Myhill-Nerode theorem), so $(L_{adv},M)$ is an effective $NT_{\forall}$- and $NT_{\exists}$-advice with regard  to $L_{dec}$.

\begin{figure}[h!]
\centering
\includegraphics[scale=0.5]{mainmatter/images/nondet.png}
\caption{a-transducer $M$}
\end{figure}

\paragraph{}
In the last example, the whole advice was in some sense contained in the transformation and the efficiency was achived just through the nondeterminism of a-transducer. We would like to prevent such a misuse of the nondeterminism, so the saving of state count would mirror the actual possibility to dismember the problem to some smaller subproblems, such that their results combined yield the solution of the whole task.

\paragraph{}
At first sight on the previous example it seems, that the problem lays in the fact, that $M$ does not have to generate the whole language $L_{adv}$. The one possible solutions is to add a condition, that the filtering of words not from $L_{dec}$ should not happen only in $M$, but also in $L_{dec}$ (and since the complexity of $L_{dec}$ is the state count of its minimal deterministic automaton, the nondeterminism could not be misused). Formally, a pair $(L_{adv}, M)$ is a ($NT_{\forall}$-) $NT_{\exists}$-advice with regard  to $L_{dec}$, if it fulfills the condition from the original definitions and moreover, we demand, that $L_{adv} \subseteq M(\Sigma_{L_{dec}}^*)$.

\paragraph{}
However, if we alter the a-transducer $M$, so that each travelsal of form $(q_i,a,a,q_j)$ will be altered to $(q_i,a,\varepsilon,q_j)$, we can take $L_{dec} = \{\varepsilon\}$. Again, $M^{-1}_{\exists}(L_{adv}) = M^{-1}_{\forall}(L_{adv}) = L_{dec}$ and the complexity of the advice has increased by $1$ (since $\C{\{\epsilon\}} = 2$), so $(L_{adv},M)$ still is an effective advice for $L_{dec}$. However, our problem with the misuse of nondeterminism is still apparent. Adding this simple condition did not help at all.

\paragraph{}
For aforementioned reason, we present the third possible definition of our framework, where the transformation model is not an a-transducer, but a (deterministic) sequential transducer. However, unlike by a-transducer, one-bounded sequential transducers are not a normal form of sequential transducers. It is easy to see, that with this restriction, the we cannot generate an output word, which is longer than the input. However, this shortcoming can be easily solved by a little alternation in a definition:
\begin{itemize}
\item $\delta$ is a partial transition function, that maps $K \times \Sigma_{1} \cup \{\varepsilon\} \rightarrow K$,
\item $\sigma$ is a partial output function, that maps $K \times \Sigma_{1} \cup \{\varepsilon\} \rightarrow \Sigma_{2}$,
\item however, the $\epsilon$-transition and $\epsilon$-output in a state $q \in K$ are possible only if there are no other transitions and outputs in this state, and
\item since $\delta$ and $\sigma$ are partial functions, we demand, that for $a \in \Sigma_1 \cup \{\varepsilon\}$ and $q \in K$, $\delta(q,a)$ is defined, if and only if $\sigma(q,a)$ is defined.
\end{itemize}

\paragraph{}
It can be easily shown, that this altered definition is a normal form of sequential transducers, however, we do not include the proof in our thesis.

\paragraph{}
\cdefinicia Let $M$ be a sequential transducer and $L$ a language. Then $M_{D}^{-1}(L)$ is the set of all words, such that their images belong to $L$. Formally \\
\centerline{$M_{D}^{-1}(L) = \{ w | M(w) \in L \}$.}

\paragraph{}
\cdefinicia Let $L_{dec}$ be a regular language. A pair $(L_{adv}, M)$, where $L_{adv}$ is a regular language and $M$ a sequential transducer is called an \emph{$T$-advice with regard to $L_{dec}$}, if there exists a deterministic finite automaton $A'$, such that $L_{dec} = L[M_{D}^{-1}(L_{adv})](A')$. Moreover, $(L_{adv}, M)$ is called \emph{effective}, if $\C{A'} + \C{M} + \C{L_{adv}} \leq	 \C{L_{dec}}$.

\paragraph{}
\cpriklad We can see, that the a-transducer $M$ from Example 2 has neither $\varepsilon$-transitions, nor multiple transitions from one state on one symbol. Moreover, the transition function is complete (the set $H$ contains an element for every combination of source state and input symbol), so the corresponding sequential transducer $M_D$ and its transition function $\delta$ and output function $\sigma$ can be easily constructed. Therefore, the pair $(L_{adv},M)$ from Example 2 is an effective $M_{D}$-advice with regard to $L_{dec}$.

\paragraph{}
\poznamka We will often use this view of a sequential transducer - as a special case of an a-transducer. When it will be suitable, we will identify a sequential transducer with an a-transducer, which set $H$ fulfills the aforementioned conditions (no $\varepsilon$-transitions and for each combination of state and input symbol precisely one element in $H$) without the formal definition of its $\delta$ and $\sigma$ functions. We state their construction here.

\centerline{$\forall h \in H: \delta(pr_0(h),pr_1(h)) = pr_3(h)$}
\centerline{$\forall h \in H: \sigma(pr_0(h),pr_1(h)) = pr_2(h)$}

\paragraph{}
The correctness of the definition of these function follows from the "determinism" of $H$.

\paragraph{}
We have defined three alternative ways to look at the use of transformation in solving problems with advisory information, which differ in the definition of the language $M^{-1}(L)$. This brings up the following question: for a given language $L$ and an a-transducer $M$, how to get the languages $M_{\forall}^{-1}(L)$, $M_{\exists}^{-1}(L)$ and $M_{D}^{-1}(L)$? The answer was quite easy to find in previous two examples (and, in fact, for all languages in form $\{ (a^k)^* \}$ and transducers, which just manipulate the number of symbols $a$). We now look at the answer in general.

\paragraph{}
\clema Let $M = (K, \Sigma_1, \Sigma_2, H, q_0, F)$ be an a-transducer and $L$ a language. Moreover, let $L' = M_{\exists}^{-1}(L)$. Then $\forall w \in L'^c: M(w) = \emptyset \vee M(w) \subseteq L^c$. The mapping $M_{\exists}^{-1}$ can be simulated by an a-transducer $M'$ dual to $M$, such that $M'(L) = L'$, where $\C{M'} = \C{M}$.

\paragraph{}
\dokaz The first part is quite easy to see, since by definition, $M_{\exists}^{-1}(L)$ contains all words, such that at least one of their images by a-transducer $M$ belongs to $L$. If for a word $v \in L'^c$ there is a word $u$, such that $u \in M(v)$ and $u \notin L^c$, then $u \in L$ and by definition, $v \in L'$, which leads to a contradiction.

\paragraph{}
We prove the second part of our Lemma constructively. Let $M' = (K, \Sigma_2, \Sigma_1, H', q_0, F)$, where\\
\centerline{$H'=\{(p,x,y,q)|(p,y,x,q) \in H\}$.} 

\paragraph{}
Clearly, $\C{M} = \C{M'}$. It remains to show, that $M'$ simulates $M_{\exists}^{-1}$, namely that $M'(L) = L'$ (since $L' = M_{\exists}^{-1}(L)$).
\begin{itemize}
\item $L' \subseteq M'(L)$: Take an arbitrary word $u \in L'$. By definition of $M_{\exists}^{-1}$, there is a word $v \in L$, such that $v \in M(u)$. Now, let us look at this computation of $M$ on $u$ as a word $h \in \Pi_M$ (see Chapter 1). Since this computation is accepting and its output is $v$, we can rewrite $h$ as a sequence of quadruples $(q_0, x_1, y_1, p_1)$ $(p_1,x_2,y_2,p_2)...$ $(p_{i-1},x_i,y_i,p_i)...$$(p_{n-1},x_n,y_n,q_F)$, where $pr_1(h) = u$, $pr_2(h) = v$ and $q_F \in F$. We now present the computation of $M'$, which shows, that $u \in M'(v)$. The computation is $h' \equiv (q_0, y_1, x_1, p_1)$$(p_1,y_2,x_2,p_2)...$ $(p_{i-1},y_i,x_i,p_i)...$$(p_{n-1},y_n,x_n,q_F)$. The plausibility of this computation follows from the construction of $M'$. We have shown, that $u \in M'(L)$ and therefore $L' \subseteq M'(L)$.
\item $M'(L) \subseteq L'$: Once again, let us take a word $u \in M'(L)$. There is a word $v \in L$, such that $u \in M'(v)$. Again, we can look at the respective computation of $M'$ on $v$ as a word $h' \equiv (q_0, y_1, x_1, p_1)$$(p_1,y_2,x_2,p_2)...$ $(p_{i-1},y_i,x_i,p_i)...$$(p_{n-1},y_n,x_n,q_F)$, where $pr_1(h') = v$ and $pr_2(h') = u$. We construct the computation $h$ of $M$ in the same way as in previous part of the proof. The computation $h$ shows, that $v \in M(u)$ and therefore $M(u) \subseteq L$ (whole $M(u)$, since all words $v$, such that $u \in M'(v)$ have to belong to $L$ according to the first part of Lemma). From the definition of $M_{\exists}^{-1}$ it follows, that $u \in M_{\exists}^{-1}(L) = L'$.
\end{itemize} \qed

\paragraph{}
Very similar result can be stated for the setting with a sequential transducer. For a sequential transducer $M$ and a language $L$, $M_{D}^{-1}(L) = M_{\exists}^{-1}(L)$, since every word $w \in M^{-1}_{D}(L)$ has exactly one image $M(w) \in L$. This means, that we can find $L$ using the same dual a-transducer $M'$. Note, that this dual machine does not necessarily be a sequential transducer, because the mapping by $M$ is not necessarily injective (and even if it is, the functions $\delta$ and $\sigma$ do not say anything about uniqeness of the combination of output symbol and resulting state). However, the determinism of the sequential transducer allows us to state some additional claims.

\paragraph{}
\clema Let $M = (K, \Sigma_1, \Sigma_2, \delta, \sigma, q_0, F)$ be a sequential transducer and $L$ a language. Moreover, let $L' = M_{D}^{-1}(L)$. Then, $M(L') \subseteq L$ and $\forall w \in L'^c: M(w) = \emptyset \vee M(w) \in L^c$. The mapping $M_D^{-1}$ can be simulated by an a-transducer $M'$ dual to the sequential transducer $M$, such that $M'(L) = L'$ and $\forall w \in L^c: M'(w) = \emptyset \vee M'(w) \subseteq L'^c$. Moreover, $\C{M'} = \C{M}$.

\paragraph{}
\dokaz We prove just that parts of our Lemma, which are different from the claims in the previous one. In the first part, we state, that $M(L') \subseteq L$. This claim follows directly from the fact, that every word $w \in L'$ has only one image $M(w)$ and by definition, this image belongs to $L$ (otherwise $w \notin L'$). Moreover, since the image of a word $w$ by a sequential transducer is a word, instead of a set, the condition on words from $L'^c$ changes accordingly.

\paragraph{}
We provide the formal construction of the a-transducer $M'$, since we construct it from the sequential transducer $M$. Again, $M' = (K,\Sigma_1,\Sigma_2,H,q_0,F)$, where \\
\centerline{$H = \{q,a,\sigma(a),\delta(q) | \forall q \in K, a \in \Sigma_1 \}$.}

\paragraph{}
The proof of the claim, that $M'(L) = L'$, is very similar to the proof of previous Lemma. We provide the arguments for the last part of Lemma, i. e. $\forall w \in L^c: M'(w) = \emptyset \vee M'(w) \subseteq L'^c$: Assume there is a word $w \in L^c$, such that $M'(w) = u \wedge u \in L'$. From previous part of Lemma it follows, that $u \in M'(L)$. However, then $w = M(u) \subseteq L$, which leads to a contradiction. \qed

\paragraph{}
We have seen, that finding the sets $M_{\exists}^{-1}(L)$ and $M_D^{-1}(L)$ for a given language $L$ and transducer $M$ is quite easy using a dual a-transducer $M'$. However, the situation with $M_{\forall}^{-1}$ is not that simple. The main problem is, that if some word $w \in \Sigma_{L'}^*$ has an image $L$, it can also have other images in $L^c$, therefore $w \notin L'$. However, if we used a dual a-transducer $M'$ from previous Lemmas on $L$, the word $w$ will be constructed, since $w \in M'(L)$. We now present the solution to this issue.

\paragraph{}
\clema Let $M = (K, \Sigma_1, \Sigma_2, H, q_0, F)$ be an a-transducer and $L$ a language. Moreover, let $L' = M_{\forall}^{-1}(L)$. Then, $M(L') \subseteq L$. The mapping $M_{\forall}^{-1}$ can be simulated by an a-transducer $M'$ dual to the sequential transducer $M$, such that $L' = M'(L) - M'(L^c)$ and $\forall w \in L^c: M'(w) = \emptyset \vee M'(w) \subseteq L'^c$. Moreover, $\C{M'} = \C{M}$.

\paragraph{}
\dokaz The first part ($M(L') \subseteq L$) follows from the definition. If a word $w$ belongs to $L'$, all of its images by $M$ are in $L$, therefore whole $M(w) \subseteq L$ and furthermore $M(L') \subseteq L$.

\paragraph{}
Now we prove the second claim in three steps, using the same construction of the dual a-transducer $M'$ as before.
\begin{itemize}
\item $L' \subseteq M'(L) - M'(L^c)$: Let $w \in L'$. By definition, $M(w) \neq \emptyset \wedge M(w) \subseteq L$, therefore, there is a word $u \in M(w) \wedge u \in L$. By the construction of $M'$, it can be easily seen (and proven similarly to the proof of the $M_{\exists}^{-1}$ case), that $w \in M'(u) \subseteq M'(L)$. Furthermore, if $w \in M'(L^c)$, it means, that there is a word $v \in L^c$, such that $w \in M'(v)$. However, then $v \in M(w)$ and since $v \notin L$, then $M(w) \not \subseteq L$ and by definition $w \notin L'$.

\item $M'(L) - M'(L^c) \subseteq L'$: Let $w \in M'(L) - M'(L^c)$. Since $w \in M'(L)$, we know, that there is at least one word $u$, such that $u \in L \cap M(w)$. The second part, i. e. $w \notin M'(L^c)$ secures, that $L^c \cap M(w) = \emptyset$ (if there was a word $u \in L^c \cap M(w)$, then $w \in M'(u) \subseteq M(L^c)$). Thus, $w$ fulfills the definition of $M^{-1}_{\forall}(L)$, therefore $w \in L'$.

\item $\forall w \in L^c: M'(w) = \emptyset \vee M'(w) \subseteq L'^c$: This claim follows directly from the fact, that $L' \cap M'(L^c) = \emptyset$.
\end{itemize} \qed

\paragraph{}
We conclude this section with a note, which will later be useful by comparing our three settings to each other. This its claim follows directly from the fact, that a sequential transducer is a special case of an a-transducer and the definition of $M^{-1}_{D}$ fulfills the definitions for both $M_{\forall}^{-1}$ and $M_{\exists}^{-1}$.

\paragraph{}
\poznamka Every (effective) $T$-advice is also a $NT_{\forall}$-advice. Every (effective) $T$-advice is a $NT_{\exists}$-advice.

\section{Decomposable and undecomposable languages}

\paragraph{}
In the previous section, we have defined the notion of an effective advice. Now we present another related concept, namely the $T$-, $NT_{\forall}-$ and $NT_{\exists}$ decomposability of regular languages.

\paragraph{}
\cdefinicia The language $L$ is called \emph{$T$-decomposable}, if there is a sequential transducer $M$ and a regular language $L_{adv}$, such that $(L_{adv}, M)$ is an effective $T$-advice for $L$. Otherwise, we call $L$ \emph{$T$-undecomposable}.

\paragraph{}
\cdefinicia Let $q \in \{\exists,\forall\}$. The language $L$ is called \emph{$NT_{q}$-decomposable}, if there is an a-transducer $M$ and a regular language $L_{adv}$, such that $(L_{adv}, M)$ is an effective $NT_q$-advice for $L$. Otherwise, we call $L$ \emph{$NT_q$-undecomposable}.

\paragraph{}
\clema Every $T$-decomposable language is $NT_{\forall}$- and $NT_{\exists}$-decomposable. Every $NT_{\forall}$- and $NT_{\exists}$-undecomposable language is $T$-undecomposable.

\paragraph{}
\dokaz Follows directly from the final remark in the previous section. \qed

\paragraph{}
Later we will see, that the reverse implication does not hold. We now compare our settings to the setting presented by \cite{Gazi} (see Section 2.3). To make the comparison more meaningful, we have to strengthen the condition presented by Gazi in a following way:

\paragraph{}
\cdefinicia A language $L$ is called \emph{$A$-decomposable}, if there exists an advisor $L_1$ and an automaton $A$, such that $\C{L_1} + \C{A_1} < \C{L}$ and $L[L_1](A) = L$.

\paragraph{}
For the sake of simplicity, we present just the comparison of $A$-decomposable and $T$-decomposable languages, since the relations to $NT_{\forall}$  and $NT_{\exists}$ follows directly.

\paragraph{}
\cveta Every $A$-decomposable language is $T$-decomposable.

\paragraph{}
\dokaz Easy to see, using an a-transducer computing the identity. \qed

\paragraph{}
However, the next theorem shows, that the reverse implication does not hold.

\paragraph{}
\cveta There are infinitely many $T$-decomposable languages, that are not $A$-decomposable.

\paragraph{}
\dokaz Such languages are for example $L_{n} = \{ a^n \}$ for $n \geq 10$ and even.

\paragraph{}
We prove this claim in two steps. First, we need to show, that $L_{n}$ is $T$-decomposable. It is easy to see, that a DFA accepting $L_{n}$ needs at least $n+2$ states, therefore $\C{L_n} = n + 2$.

\paragraph{}
However, we can use an advice to simplify the accepting automaton as follows: our sequential transducer $M$ will encode each pair of letters $a$ into one new letter $b$ using two states, where the first state is accepting. Formally, $M = (\{q_0,q_1\},\{a\},\{b\},\delta,\sigma,q_0,\{q_0\})$, where\\
\centerline{$\delta(q_0,a) = q_1; \delta(q_1,a)=q_0$, and}
\centerline{$\sigma(q_0,a) = b; \sigma(q_1,a) = \varepsilon$.}

\paragraph{}
Now, the advisory language is $L_{n,adv} = \{ b^{\frac{n}{2}} \}$, while $\C{L_{n,adv}} \leq \frac{n}{2} + 2$.

\paragraph{}
We need to construct an automaton $A$, such that $L[M^{-1}_{D}(L_{n,adv})](A) = L_n$. Let $L(A) = \{a\}^*$. Clearly, $M^{-1}_{D}(L_{n,adv}) = L_n$, so the advice gives the full information about $L_n$. Altogether, we used $2 + \frac{n}{2}+2+1$ states, therefore for $n \geq 10$ is $(L_{n,adv}, M)$ an effective $T$-advice with regard to $L_n$.

\paragraph{}
Our next goal is to show, that $L_n$ is not $A$-decomposable. As we have said before, a minimal DFA $A$ for $L_n$ has $n+2$ states and its states correspond to the equivalence classes of the relation defined by Myhill-Nerode theorem (see Section 2.2.1). These equivalence classes are:
\begin{enumerate}
\item $[c_0] = \{ \varepsilon \}$,
\item $[c_{i}] = \{ a^i \}$ for $1 \leq i \leq n$,
\item $[c_{n+1}] = \{ a^k | k > n \}$.
\end{enumerate}

\paragraph{}
We proceed by contradiction, therefore we assume, that we can find an automaton $A'$ and a language $L_{adv}$ (with an automaton $A_{adv}$), such that $\C{A'} + \C{L_{adv}} < \C{L_n} = n+2$ and $L[L_{adv}](A') = L_n$. We will show, that both $A'$ and $A_{adv}$ need at least $n$ states, otherwise they would accept an input from $[c_{n+1}]$, which leads to a contradiction, since $\C{A'} + \C{L_{adv}} \geq n+n \geq n+2 = \C{L_n}$.

\paragraph{}
Let us now look at the minimal deterministic finite  automaton $A_{adv}$ of $L_{adv}$. Since the inequality holds, $A_{adv}$ has at most $n$ states. Also, $A_{adv}$ accepts the language $L_n$, that means, in our case, the word $a^n$. Clearly, by reading $a^n$, $A_{adv}$ runs in a cycle. Without loss of generality, assume that in one iteration of the shortest cycle $A_{adv}$ reads $a^l$. Therefore, it accepts also incorrect outputs in form of $a^{n + s.l}, s \geq 1$.

\paragraph{}
The same argument can be used for $A'$. Assume, that it accepts also words $a^{n + s.k}, s \geq 1$. However, this means, that $a^{n+s.k.l} \in L_{adv}$ and also $a^{n+s.k.l} \in L(A')$ and our model accepts the word $a^{n + s.k.l}$. However, $a^{n + s.k.l} \notin L_n$.\qed

\paragraph{}
\cdosledok There are infinitely many $T$-decomposable languages.

\paragraph{}
\cdosledok There are infinitely many $NT_{\forall}$- and $NT_{\exists}$-decomposable languages.

\paragraph{}
We have seen, that adding the possibility of transformation in solving problems with supplementary information can help us to also decompose some languages, that are not decomposable without the use of transformation. Further we show, that also adding the possibility to use nondeterminism in the transformation gives us more power (i. e. the settings which use nondeterministic transformation yield bigger classes of decomposable languages).

\paragraph{}
\cveta There are languages, that are $T$-undecomposable, but $NT_{\exists}$- and $NT_{\forall}$-decomposable.

\paragraph{}
\dokaz An example of such a language is $L = (\{a^{2k}|k\geq 0\} \cup \{a^{13k}|k\geq 0\}) \setminus \{a^{26k}|k\geq 0\}$. Clearly, $\C{L} = 26$.

\paragraph{}
We first show the $NT_{\forall}$-decomposability and $NT_{\forall}$-decomposability of $L$. The effective advice for this language is $(L_{adv},M)$, where $M$ is the a-transducer from Figure 4.3 and $L_{adv} = \{a,b\}^* \setminus \{b^2\}^*$. For a word $a^k$ the a-transducer $M$ can generate following images:

\begin{itemize}
\item if $k = 0\ (mod\ 2)$, then $M(a^k) \ni a^{\frac{k}{2}}$
\item if $k = 0\ (mod\ 13)$, then $M(a^k) \ni a^{\frac{k}{13}}$
\item if $k = 0\ (mod\ 26)$, then $M(a^k) = \{a^{\frac{k}{2}}, a^{\frac{k}{13}}\}$
\item otherwise, $M(a^k) = \emptyset$.
\end{itemize}

\begin{figure}[h!]
\centering
\includegraphics[scale=0.5]{mainmatter/images/v-e.png}
\caption{a-transducer $M$}
\end{figure}

\paragraph{}
From above mentioned it follows, that $M^{-1}_{\forall}(L_{adv}) = L$.

\paragraph{}
Now the $NT_{\forall}$-decomposability of $L$ is easy to see, since the $NT_{\forall}$-advice gives full information about $L$ and $\C{M} + \C{L_{adv}} + \C{\{a\}^*} = 16 + 5 + 1 < 26 = \C{L}$.

\paragraph{}
It remains to show, that there is no effective $T$-advice for $L$. We will examine three elements - a sequential transducer $M'$, an advisory language $L'_{adv}$ and an automaton $A_{simple}$ with a language $L_{simple} = L(A_{simple})$, such that $L[M_{D}^{'-1}(L'_{adv})](A_{simple}) = L$.

\paragraph{}
We claim, that $L_{simple}$ contains all words of the form $a^{26k}, k \geq 1$. Assume this is not the case. Since $\C{A_{simple}} < 26$, the computation on a word $a^{26}$ contains a cycle of length $r < 26$. If $r = 13$, automaton $A$ does not accept a word $a^{13}$. If $r \neq 13$, $A$ does not accept a word $a^{26+2.r}$, while this word belongs to $L$ (it has a form $a^{2k}$). We remind, that also $L_{simple} \supseteq L$.

\paragraph{}
We now use the fact, that $\C{M'} + \C{L'_{adv}} < 26$, therefore $\C{M'}<13 \vee \C{L'_{adv}} < 13$. We examine these two cases separately.

\begin{itemize}
\item $\C{M'} < 13$: We claim, that all words from $\{a\}^*$ longer than $12$ symbols have an accepting computation on $M'$. Assume this is not the case and there is a word $a^k, k \geq 13$, such that the computation of $M'$ on $a^k$ is not accepting. Note, that $a^k \notin L$, therefore $k$ is either a multiple of $26$, or an odd number not divisible by $13$. However, this computation contains a cycle in the sequence of states. Let the input read in this cycle be $a^r$. We know, that $r < 13$. Now we distinguish two main cases:

\begin{enumerate}
\item $26 \mid k$: The computation of $M'$ on $a^{k+2.r}$ is not accepting, because $M'$ is determinisitic and this computation just contains two more iterations of aforementioned cycle, but the final state is the same. Therefore, for any $L'_{adv}$, $M^{'-1}_{D}(L'_{adv})$ does not contain $a^{k+2.r}$. However, $a^{k+2.r} \in L$, because since $r < 13$, $2.r$ surely is not a multiple of $26$. 

\item $k$ is even $\wedge 13 \nmid k$: We can again find a word from $L$, which will also not be transduced by $M'$. However, we need to differentiate, whether $r$ is odd or even.

\begin{enumerate}
\item $r$ is odd: Both words $a^{k+r}$ and $a^{k+3r}$ have a non-accepting computation on $M'$. However, since, $k$ is also odd, both of these words have even length, while at least one of them belongs to $L$, since $26 \nmid 2r$ for $r < 13$.

\item $r$ is even: Let $t = 13 - k \mod 26$. Since $k$ is even, also $t$ is even. Now, let $l$ be the smallest number, such that $l.r = t \pmod{26}$. Such a number $l$ exists, since $r < 13$ is a generator in the subgroup of even numbers in $\Z_{26}$. Now, for similar reasons as in previous case, the word $a^{k+l.r}$ does not have an accepting comutation on $M'$, while it belongs to $L$.
\end{enumerate}
\end{enumerate}

\paragraph{}
As we have shown, every word from $\{a\}^*$ longer than $12$ symbols has an image $M'(a)$. Moreover, $L_{simple}$ contains all words from $L$ and also all words of the form $a^{26m}, m \geq 1$. This leads us to a claim, that $M_D^{'-1}(L'_{adv})$ has at least $26$ equivalence classes from Myhill-Nerode theorem. Though, $\C{L'_{adv}}$, therefore there are two words $a^k, a^l$ for $k,l > 26$, such that $[a^k]_{M_D^{'-1}(L'_{adv})} \neq [a^l]_{M_D^{'-1}(L'_{adv})}$, but $[M'(a^k)]_{L'_{adv}} = [M'(a^l)]_{L'_{adv}}$. Of course, $a^k \in M_D^{'-1}(L'_{adv}) \Leftrightarrow  a^l \in M_D^{'-1}(L'_{adv})$.

\paragraph{}
We will show, that there is a suffix $a^s$, such that after adding this suffix to $a^k$ and $a^l$, one of these words belongs to $L$, while the other does not. However, the computation of $M'$ on these words runs in a 

Once again, we distinguish few cases:

\begin{enumerate}
\item $k,l$ are even, $13 \nmid k,l$: 
\end{enumerate}

\item $\C{L'_{adv}} < 13$:
\end{itemize} \qed

\paragraph{}
A natural question arises, what is the relation between the two nondeterministic concepts - in other words, is it really necessary to have two separate definitions? The next theorem gives an answer to these questions.

\paragraph{}
\cveta There are infinitely many languages, that are both $NT_{\forall}$-undecomposable and $NT_{\exists}$-undecomposable.

\paragraph{}
\dokaz Each of the languages $L_{p} = \{ (a^p)^* | p$ is a prime number $\}$ is T-undecomposable.

\paragraph{}
\color{red}prerobit\color{black}
It is easy to see, that $\C{L_p} = p$. Let us fix a particular $p$ (the arguments will work for all prime numbers $p$, we fix it in order to simplify the notation). We want to decompose $L_p$ to get a simpler automaton $A'$. Let $L_{simple} = L(A')$. Moreover, we will be looking for an advisory language $L_{adv}$ and an a-transducer $M$. Let $L_{trans} = M^{-1}(L_{adv})$.

\paragraph{}
Now, we present some constraints on aforementioned languages. From the definition of the framework, we know, that $L[L_{trans}](A') = L_p$ and therefore $L_p = L_{simple} \cap L_{trans}$. We claim, that $\C{L_{simple}} \geq p$ or $\C{L_{trans}} \geq p$. This can be proven using a series of arguments, which have been already used couple of times in our thesis - since both languages must contain $L_p$ as their subset, if both finite automata have fewer than $p$ states, their computation would run in a cycle of some lengths $k, l$. Then, both automata would accept some word extended by a suitable common multiple of $k$ and $l$ symbols $a$, which however does not belong to $L_p$.

\paragraph{}
On the other side, since we claim, that $L_p$ is T-decomposable, it must hold, that $\C{L_{simple}} \allowbreak < p-1$ (together with another two devices, the total number of states is at most $p$). It follows, that $\C{L_{trans}} \geq p$. What do we know about the complexity of $L_{adv}$? For similar reasons as for $L_{simple}$, also for $L_{adv}$ it has to hold, that $\C{L_{adv}} < p-1$.

\paragraph{}
That means, that in fact, we want to encode the language $L_{trans}$ into the language $L_{adv}$ with a smaller complexity using an a-transducer $M$. However, we not only need, that $M(L_{trans}) = L_{adv}$. Lemma 14 gives us another supplementary conditions on $M$. We want to find two dual a-transducers $M, M'$, such that $M(L_{adv}) = L_{trans}$, $M'(L_{trans}) = L_{adv}$ and both $\C{M} < p$ and $\C{M'} < p$. Besides, $M(L^c_{adv}) \cap L_{trans} = \emptyset$ and $M'(L^c_{trans}) \cap L_{adv} = \emptyset$ (this is just another notation of conditions from Lemma 14).

\paragraph{}
Now, let us consider an a-transducer $M'$ with aforementioned properties and the language $M'(L_{trans})$. We know, that $L_{trans}$ contains all words of the form $(a^p)^*$ and all this words have to be transduced by $M'$ to words from $L_{adv} (M'(L_{trans}))$. Since $M'$ has fewer than $p$ states, the computation of $M'$ on such words contains a cycle. Without loss of generality, take one of the accepting computations on $a^p$ and let the shortest cycle with output have length $c$ (if no cycle has an output, \color{red}lepsi argument je, ze by sme nevedeli generovat dlhsie vystupy\color{black} $M'(a^p) = M'(a^{p+c})$, which violates the condition in Lemma 14). Let the output of this cycle be $x \neq \varepsilon$. Moreover, assume, that the word generated by this computation is $w$.

\paragraph{}
Now, we know, that $w \in L_{adv}$. $M'$ gives correct outputs on all words from $L_{trans}$, so also if we iterate the cycle few more times, we get to the same accepting state.  The substring $x$ can be anywhere in the word $w$, however, without loss of generality, we may assume, that it is in the end. Otherwise the arguments are the same, but with slithly more complex notation. So, we assume $w.(x^{k.p}) \in L_{adv}$ for any $k \geq 0$ (because $w.(x^{k.p}) \in M'(a^{p+k.p.c})$). Consider words of the form $u_i = a^{p+i.c}$ and $v_i = w.(x^i)$ for all $i > 0$. We know, that $v_i \in M'(u_i)$. Using the equivalence relation from Myhill-Nerode theorem (see Section 2), we show, that the number of equivalence classes of $L_{adv}$ is bigger than $p$, which contradicts our assumptions.

\paragraph{}
We claim, that each of the words $v_i$ for $1 \leq i \leq p$ yields another equivalence class. Assume there is a pair of indices $k,l; 1 \leq k < l \leq p$, such that $v_k$ and $v_l$ fall into the same equivalence class. Let $m$ be the smallest number such, that $m>p \wedge v_m \notin L_{adv}$.

\paragraph{}
We will call a number $n$ "bad", if $v_k.(x^{n}) \notin L_{adv}$. We know, that $m-k$ and $m-l$ are bad. For the sake of simplicity, let $r = l-k$. Since $r < p$, at least one of $m-k, m-l$ is not a multiple of $p$. Without loss of generality, let it be $m-k = s$ (the other case is very similar). $v_k$ and $v_l$ are in the same equivalence class, so if $s$ is bad, also $s+r$ is bad. For the same reason, if $s+r$ is bad, also $s+2r$ is bad and all $s+t.r$ for $t \geq 0$ are bad. However, from the group theory we know, that $\Z_p$ is a cyclic group, where every $i \neq 0$ is a generator. Therefore, also $r$ is a generator and for some $j$, $j.r$ is the inverse element to $s+k$. However, this would mean, that $v_{j.r+(s+k)} \notin L_{adv}$, which further means, that $u_{j.r+(s+k)} \in L_{trans}$, while $j.r+(s+k)$ is divisible by $p$, which leads to a contradiction.

\paragraph{}
We need to examine one additional case - if $\forall i, p \leq i: v_i \in L_{adv}$. This means, that $\forall i, p\leq i: u_i = a^{p+i.c} \in L_{trans}$. Though, we have seen, that the automaton $A'$ for $L_{simple}$ has less than $p$ states and again, it runs in a cycle of length $d < p$ when accepting words from $L_{dec}$. Hence a word $a^{p+d.c} \in L_{simple}$. As we have seen, also $a^{p+d.c} \in L_{trans}$, but then $a^{p+d.c} \in L_p$, which is a contradiction, because both $d,c<p$ and $p$ is a prime number. \qed

\paragraph{}
As we have seen, the classes of regular languages concerning T-decomposability are different as the classes of A-decomposable and A-undecomposable languages. In the next part of our thesis, we investigate some properties of these classes.

%------------------------------------------------------------------------------------
\section{Closure properties}

\paragraph{}
When looking at a new class of languages, one of the first natural question, that arises, are its closure properties.  In this section, we examine the closure of T-decomposable and T-undecomposable languages under some basic operations and then under deterministic operations presented in \cite{AFDL}.

\subsection{T-undecomposable languages}
\paragraph{}
In this part, we mainly use two types of T-undecomposable languages. First of them are languages of type $L_p = \{ a^{pk} | k \geq 0 \}$ for $p$ a prime number. The T-undecomposability of these languages is proved in previous Section. The second type is a language $L = \{ a \}^*$. This language is clearly undecomposable, since $\C{L} = 1$ and all three devices contained in our foreign advisor concept have non-zero number of states.

\paragraph{}
\cveta The class of T-undecomposable languages is not closed under 
\begin{enumerate}
\item complement,
\item (non-erasing) homomorphism,
\item inverse homomorphism,
\item Kleene star, Kleene plus,
\item intersection,
\item union.
\end{enumerate}

\paragraph{}
\dokaz
\begin{enumerate}
\item \color{red}najst schodny dokaz, posledne dva nevysli\color{black}

\item Consider an undecomposable language $L_1 = \{ a^{13k} | k \geq 0 \}$ and a homomorphism $h:\{ a\}^* \to \{ a \}^*$, such that $h(a) = aa$. Clearly, $h(L_1) = \{ a^{26k} | k \geq 0 \}$ and this language can be decomposed in a following way: let us take an a-transducer $M_1$ computing the identity mapping and a language $L'_1 = \{ a^{2k} | k \geq 0 \}$. This two items form the desired effective advice for $L_1$, since we only have to chceck the language $\{ a^{13k} | k \geq 0 \}$.

Since this homomorphism is non-erasing, our class is not closed even under this kind of mapping.

\item \color{red}ToDo: najst nejaky schopny protipriklad\color{black}

\item \color{red}ToDo: dobra otazka, mozno nakoniec aj bude - ak ma jazyk  tvar $L^*$, potom asi musi byt v automate prechod akceptacny -> pociatocny a tu to vieme roztrhnut a potom ak sa dal zjednodusit ten cely, tak sa musi dat aj ten maly. lenze, tazko povedat, mozno sa ten jazyk tak nejak moze zvrhnut, ze sa to zrazu bude dat rozkladat. zistit!\color{black}

\item Consider two languages, $L_{41} = \{ a^{13k} | k \geq 1 \}$ and $L_{42} = \{ a^{2k} | k \geq 1 \}$. As stated before, both of these languages are T-undecomposable. However, $L_{41} \cap L_{42} = \{ a^{26k} | k \geq 1 \}$ is a T-decomposable language, as we have seen in the first part of this proof.

\item Let us take two languages, $L_{51} = \{ a^k | k \neq 0 (mod 5) \}$ and $L_{52} = \{ a^k | k \neq 0 (mod 7) \}$. The T-undecomposability of these languages can be shown in very similar way, which we have seen in Theorem 17. We show just the first part of the proof, since the rest follows the same pattern as in aforementioned result.

\paragraph{}
Assume, that $L_{51}$ is T-decomposable (the same train of thoughts can be used for $L_{52}$). This means, that we can decompose $L_{51}$ into two languages $L_{trans}$ and $L_{simple}$, such that $L_{trans} \cap L_{simple} = L_{51}$. As we see, $L_{51} \subseteq L_{simple}$. However, since the finite automaton for $L_{51}$ has fewer than $5$ states, it is easy to see, that $L_{simple} \supseteq \{a^*\}$ (otherwise at least one of the words $a, aa, aaa, aaaa$ would be rejected). It  follows, that $L_{trans} \cap \{ a^k | k = 0 (mod 5) \} = \emptyset$. As we  have said, the rest of the proof is almost identical to the proof of Theorem 15.

\paragraph{}
Now we claim, that the language $L_{53} = L_{51} \cup L_{52} = \{a^k|k\neq 0 (\mod 35)\}$ is T-decomposable. Clearly, $\C{L_{53}} = 35$. Take the a-transducer $M_5$ from Figure 4.2. It can be seen, that $M_5(a^{5k} = \{ b^k \}$ and $\forall k \neq 5: M_5(a^k) = \{ a^k \}$. Now, let $L_{adv} = \{ w \in \{a,b\}^* | \nexists i: w = b^i \wedge i = 0 (mod 7) \}$. Then, $L_{simple} = \{a\}^*$ and clearly, $(L_{adv},M_5)$ is an effective advice with regard to $L_{53}$.

\end{enumerate} \qed

\subsection{T-decomposable languages}

\paragraph{}
\cveta The class of T-decomposable languages is not closed under 
\begin{enumerate}
\item complement,
\item (non-erasing) homomorphism,
\item inverse homomorphism,
\item Kleene star, Kleene plus,
\item intersection,
\item union.
\end{enumerate}

\paragraph{}
\dokaz
\begin{enumerate}
\item This claim follows directly from previous Theorem.

\item Let us take a language $L_1 = \{w|w \in \{ a,b\}^* \wedge \#_{a}(w) \mod 42 \equiv 0 \}$. Clearly, a language $L'_1 = \{w|w \in \{ a,b\}^* \wedge \#_{a}(w) \mod 14 \equiv 0 \}$ with an a-transducer $M_1$ computing the identity mapping is an effective advice for $L_1$.

Let us now consider a homomorphism $h: \{ a,b\}^* \to \{ a \}^*$, defined as $h(a) = a, h(b) = a$. Note, that $h$ is a non-erasing homomorphism. It easy to see, that $h(L_1) = \{ a \}^*$, however, as stated before, this language is T-undecomposable.

\item Consider a language $L_2 = \{ a^{26k} | k \geq 1 \}$. The decomposition of this language was shown in the proof of previous theorem. The desired homomorphism is $h: \{a\}^* \to \{a\}^*$, where $h(a) = aa$. Now, $h^{-1}(L_2) = \{ a^{13k} | k \geq 1 \}$, which is T-undecomposable.

\item \color{red}prerobit\color{black}

The counterexample is a language $L_3 = \{ a^{11} \}$. Let us take a language $L'_3 = \{ a^{5} \}$; an a-transducer $M_3 = (\{q_0, q_1, q_2\}, \{a\}, \{a\}, H, q_0, \{q_1\})$, where $H = \{ (q_0, a, \varepsilon, q_1), (q_1, a, \varepsilon, q_2),\allowbreak (q_2, a, a, q_1) \}$; and an automaton $A_3 = (\{q_0\}, \{a\}, \delta, q_0, \{q_0\}) $, where $\delta(q_0, a) = q_0$. Clearly, $M_3^{-1}(L'_3) = L_3$ and $\C{L'_3} + \C{T} + \C{A_3} = 5 + 3 + 1 \leq 9 = \C{L_3}$, therefore $L_3$ is T-decomposable. Though, $(L_3)^+ = \{ a^{11k} | k \geq 1 \}$ and $(L_3)^* = \{ a^{11k} | k \geq 0 \}$ are T-undecomposable.

\item Let us take a look at two languages, $L_{41} = \{ a \}^* \cup \{ b^{15k} | k \geq 1 \}$ and $L_{42} = \{ a\}^* \cup \{ c^{15k} | k \geq 1 \}$. We show the decomposition of $L_{41}$, since that of $L_{42}$ is very similar.

Let $L'_{41} = \{ a\}^* \cup \{ b^{3k} | k \geq 1 \}$ and $M_{41}$ compute the identity mapping. With this advice, we need to check just the language $L''_{41} = \{ a\}^* \cup \{ b^{5k} | k \geq 1 \}$ with an automaton $A''_{41}$. It is easy to see (and provable by Myhill-Nerode theorem), that a DFA for language $L_{41}$ needs at least $18$ states. However, $\C{L'_{41}} + \C{M} + \C{L''_{41}} = 6 + 1 + 8 = 15$ and clearly $L[L'_{41}](A''_{41}) = L_{41}$, which means, that $L_{41}$ is T-decomposable.

However, if we take the language $L_4 = L_{41} \cap L_{42} = \{ a\}^*$, we get a T-undecomposable language, therefore our class is not closed under intersection.

\item In previous section we have seen, that languages $L_{61} = \{a^{10}\}$ and $L_{62}=\{a^{12}\}$ are T-decomposable. Now we show, that also their complements are T-decomposable. Let $M_6 =(\{q_0,q_1,q_2\}, \{a\},\{a,b\},H,q_0,\{q_0,q_2\})$, where $H = \{(q_0,a,\varepsilon,q_1),\allowbreak (q_1,a,b,q_0),\allowbreak (q_0,a,a,q_2)\}$. It can be easily seen, that $M(a^{2k}) = b^k$ and $M(a^{2k+1}) = b^ka$. Now, the effective advice for $L_{61}^c$ consits of $M$ and $L_{61,adv} = \{b^5\}^c$. Clearly, $\C{M} + \C{L_{61,adv}} + \C{\{a\}^*} = 3+7+1 \leq 12 = \C{L_{61}^c}$ and $L_{61}^c = M^{-1}(L_{61,adv}) \cap \{a\}^*$. The effective advice for $L_{62}^c$ can be constructed in the same way.

However $L_{61}^c \cup L_{62}^c = \{a\}^*$ and since $\C{\{a\}^*} = 1$, this language is T-undecomposable.
\end{enumerate} \qed
