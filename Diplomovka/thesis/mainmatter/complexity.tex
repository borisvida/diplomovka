\chapter{Complexity of a-transducers}
\label{chap:complexity}

\paragraph{}
This section is concerned with the complexity of a-transducers. Since the majority of the results published to this date involve sequential transducers and sequential functions, we try to investigate two new concepts in this area - nondeterminism and the fact, that we deal with pairs of languages, without exactly defined transduction.

\paragraph{}
However, at this time we do not have any universal way of proving the minimality of an a-transducer (in regard to number of states). For this reason, we would like to look at some special classes of transformations and languages and present the results concerning these.

\paragraph{}
As we know, for a regular language $R$, there always exists an a-transducer with $\C{R}$ states, which generates $R$ "from scratch", regardless of the input - we take the finite automaton for $R$ and alter its transition function from reading to generating symbols.

\paragraph{}
Formally, for an automaton $A = (K, \Sigma, \delta, q_0, F)$ we can construct an a-transducer $M = (K, \Sigma, \Sigma, H, q_0, F)$, where $H = \{ (p, \varepsilon, a, q) | \delta (p, a) = q\} \cup \{ (q_0, a, \varepsilon, q_0) | a \in \Sigma \} $. We can look at the computation of $A$ as a sequence of pairs $(q, a)$, where in each step, $A$ is in the state $q$ and reads symbol $a$. The a-transducer $M$ will work in the same way, except instead of reading symbol $a$, $M$ reads in each step $\varepsilon$ and writes $a$ on the output. Then, in the final state, $M$ consumes the whole input without generating any output. It is easy to see, that for any nonempty language $L$, $M(L) = R$.


%----------------------------------------------------------------------------
\section{"Modular-counting" languages}

\paragraph{}
By "modular counting" languages we understand languages in the form \\
\centerline{$L_k = \{ a^k | k \equiv 0 (mod k) \} $.}

\paragraph{}
We would now like to present our results concerning the minimum complexity of an a-transducer for a pair of modular counting languages.

\paragraph{}
\oznacenie By $\gcd(k,l)$ we denote a greatest common divisor of integers $k,l$, by $\lcm(k,l)$ their lowest common multiple.

\paragraph{}
\clema For a pair of languages $L_k, L_l$, the minimal state complexity of an a-transudcer $M$, such that $M(L_k) = L_l$, is 
\begin{enumerate}
\item $l$, if $k$ and $l$ are coprime integers,
\item $\frac{l}{\gcd(k,l)}$, if $k \leq l$,
\item $\min(l,\frac{k}{\gcd(k,l)})$, if $l < k < l^2$,
\item $l$, if $k \geq l^2$.
\end{enumerate}

\paragraph{}
\dokaz For the sake of clarity, we prove the four parts of the Lemma separately. However, as stated before, $l$ states are always sufficient, so we have a natural upper bound for parts 1. and 4.
\begin{enumerate}
\item Let $M = (K, \{a\}, \{a\}, H, q_0, F)$ be an a-transducer, such that $M(L_k) = L_l$. Let $M$ have $l-1$ states. Now, let us look at an accepting computation (in this case the sequence of states) of $M$ on some sufficiently long word $x \in L_k$ ($|x| \geq l$), on which $M$ generates a word $y \in L_l$. Clearly, there has to be a cycle, i. e. the computation has a form $q_0, q_1, ..., q_i, ..., q_j, ..., q_F$, where $q_F \in F$ and $q_i \equiv q_j$, while $j < i + l$ (we assume that this is the shortest cycle in the computation, during that $M$ generates a non-empty output). In this cycle, $M$ reads a subword $a^r$ and generates output $a^s$ for some $r,s; 1 \leq r,s \leq l-1$.

\paragraph{}
Now, let us take two longer inputs $x' \equiv x.a^{k.r}$ and $x'' \equiv x.a^{2k.r}$. On these two inputs, $M$ generates outputs $y' \equiv y.a^{k.s}$ and $y'' \equiv y.a^{2k.s}$, respectively. Since $k$ and $l$ are coprime integers and $s < l$, $k.s$ is not divisible by $l$ (the least common multiple of two coprimes is their product), therefore at least one of these outputs does not belong to $L_l$, while both $x', x'' \in L_k$. We have generated an incorrect output, thus $M$ cannot have less than $l$ states.

\item As claimed before, an a-transducer with $l$ states does the job. Therefore, in further we assume, that $\frac{l}{\gcd(k,l)} < l$.

\paragraph{}
First we will show, that $\frac{l}{\gcd(k,l)}$ states suffice. We can construct an a-transducer $M = (K, \{ a\}, \{ a\}, H, q_0, F)$, where
\begin{itemize}
\item $K = \{ q_0, q_1,  ..., q_{\frac{l}{\gcd(k,l)}-1 }\}$
\item $F = q_0$%q_{\frac{l}{\gcd(k,l)}-1 }$
\item $H = \{(q_i, a, a, q_{i+1})| 0 \leq i < \frac{k}{\gcd(k,l)}-1 \} \cup \{(q_i, \varepsilon, a, q_{i+1})| \frac{k}{\gcd(k,l)}-1 \leq i < \frac{l}{\gcd(k,l)}-2 \} \cup \{ (q_{\frac{l}{\gcd(k,l)}-1}, \varepsilon, a, q_0) \}$.
\end{itemize}

\paragraph{}
It is easy to see, that the number of iterations of this cycle on a correct input (from $L_k$) is divisible by $\gcd(k,l)$. Each iteration creates $\frac{l}{\gcd(k,l)}$ symbols $a$ on the output, therefore $M(L_k) = L_l$.

\paragraph{}
Now we need to prove, that this number really forms a lower bound for state count: suppose, that there is an a-transducer $M' = (K, \{a\}, \{a\}, H, q_0, F)$ with at most $\frac{l}{\gcd(k,l)} -1$ states. Similarly to the proof of part 1., we look for a cycle, in this case of the length $\frac{l}{\gcd(k,l)} - 1$ states. With very similar series of arguments, we can construct two inputs $x' \equiv x.a^{k.r}$ and $x'' \equiv x.a^{2k.r}$, which produce outputs $y' \equiv y.a^{k.s}$ and $y'' \equiv y.a^{2k.s}$, respectively. If both of these numbers were divisible by $l$, then also $k.s$ would be divisible by $l$. However, this is not possible, since $s < \frac{l}{\gcd(k,l)}$ and as stated in Chapter 1 \color{red}(probably)\color{black}, $\lcm(k,l) = \frac{k.l}{\gcd(k,l)}$.  

\item Just like in part 2., we show, that if $k > l \land k < l^2$, then $\frac{k}{\gcd(k,l)}$ states is enough. The corresponding a-transducer will look as follows: $M = (K, \{ a\}, \{ a\}, H, q_0, F)$, where 

\begin{itemize}
\item $K = \{ q_0, q_1,  ..., q_{\frac{k}{\gcd(k,l)}-1 }\}$
\item $F = q_0$
\item $H = \{(q_i, a, a, q_{i+1})| 0 \leq i < \frac{l}{\gcd(k,l)}-1 \} \cup \{(q_i, a, \varepsilon, q_{i+1})| \frac{l}{\gcd(k,l)}-1 \leq i < \frac{k}{\gcd(k,l)}-2 \} \cup \{ (q_{\frac{k}{\gcd(k,l)}-1}, \varepsilon, a, q_0) \}$.
\end{itemize}
\paragraph{}
For similar reason as in part 2., it is clear, that $M(L_k) = L_l$.

\paragraph{}
However, the second part of the proof is a little bit different. We will not show, that an a-trandsucer $M' = (K', \{a\}, \{a\}, H', q'_0, F')$ with fewer states generates an incorrect output, but we claim, that it is not able to generate all correct outputs (i. e. all outputs from language $L_l$).
What is the shortest nonempty  word, that we can generate from $L_k$ using $M'$?

\paragraph{}
We have assumed, that $k < l^2$, therefore we can also state, that $\frac{k}{\gcd(k,l)} < l$. Once again, we look for a cycle in the computation of $M'$. Since $|Q'| < l$, to produce an output of length $l$ the computation must have a form $q'_0, q'_1, ...,$ $ q'_i, ..., q'_j, ..., q'_F$, where $q'_F \in F'$ and $q'_i \equiv q'_j$, while $j < i + \frac{k}{\gcd(k,l)}$. In each iteration of this cycle, $M'$ has to output at least one symbol $a$.

\paragraph{}
We claim, that in each iterarion of the cycle (i. e. in any of all possible cycles in its computation), $M'$ has to generate at least $\frac{l}{\gcd(k,l)}$ symbols $a$. Really, in the proof of the second part of our Lemma we have seen, that if the number $s$ - the number of output symbols generated in one iteration of the cycle - is smaller than $\frac{l}{gcd(k,l)}$, $M'(L_k) \cap L_l^c \neq \emptyset$, which leads to a contradiction.

\paragraph{}
Moreover, since $|Q'| < \frac{k}{\gcd(k,l)}$, we also know, that in one iteration of each cycle, $M'$ reads less than $\frac{k}{\gcd(k,l)}$ symbols. Now, the shortest nonempty word from $L_k$ (if $M'(\varepsilon) \neq \emptyset$, it could be trivially proven, that $M'(L_k)$ contains also words not from $L_l$) is $a^k$. The total number of iterations of all cycles is hence more than $\frac{k}{\frac{k}{\gcd(k,l)}} = \gcd(k,l)$. However, as we have claimed, every cycle generates at least $\frac{l}{\gcd(k,l)}$ symbols. Then, the smallest output length $n > \gcd(k,l).\frac{l}{\gcd(k,l)} = l$, hence we have no way to generate word $a^l \in L_l$.

\item The correctness of the lower bound $l$ is clear from the construction based on its final automaton (see above). The impossibility of existence of a smaller a-transducer follows directly from previous part of Lemma - if $k \geq l^2$, then $l \leq \frac{k}{\gcd(k,l)}$.

\end{enumerate} \qed

\paragraph{}
\cveta We can summarize previous lemmas in following claim: \\
\centerline{$\C{L_k, L_l} = \min (l, \frac{\max (k,l)}{\gcd (k,l)})$}.

%----------------------------------------------------------------------------
\section{Common transformations}

\paragraph{}
\color{red}ToDo: dovodit, preco je to uzitocne (vyuzijeme pri advisors, hopefully)\color{black}

\paragraph{}
\color{magenta}ToDo: nejak rozumne formulovat vysledok toho tvaru, ze na zmenu abecedy z k-arnej na l-arnu treba nejaky logaritmus stavov\color{black}

\paragraph{}
\color{magenta}ToDo: vysledky pri pocitani XOR s nejakym specific klucom\color{black}
