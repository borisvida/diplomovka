\chapter{Preliminaries}
\label{chap:Preliminaries}

\paragraph{}
In this section, we present some basic notation and terminology used in our thesis.

\section{Basic concepts and notation}

\paragraph{}
\oznacenie In our thesis we use the following notation: $\epsilon $ denotes an empty string, $|w|$ the length of a word $w$ ($|\epsilon |=0$), $|A|$ the number of elements of a finite set (or a finite language) $A$, $\# _{a}(w)$ the number of occurences of the symbol $a$ in the word $w$, $2^{A}$ the set of all subsets of $A$.

\paragraph{}
\definicia A \emph{homomorphism} is a function $h: \Sigma_{1}^{*} \rightarrow \Sigma_{2}^{*}$, such that \\	
\centerline{$\forall u, v \in \Sigma_1^*: h(uv) = h(u)h(v)$}

\paragraph{}
\oznacenie If $\forall w \neq \epsilon : h(w) \neq \epsilon $, we call $h$ an \emph{$\epsilon $-free homomorphism}. Usually, we denote an $\epsilon $-free homomorphism by $h_{\epsilon }$.

\paragraph{}
\definicia An \emph{inverse homomorphism} is a function $h^{-1}: \Sigma_{1}^{*} \rightarrow 2^{\Sigma_{2}^{*}}$, such that $h$ is a homomorphism and \\
\centerline{$\forall u \in \Sigma_1^*: h^{-1}(u) = \{ v \in \Sigma_2^* | h(v) = u \} $}

\paragraph{}
\definicia A \emph{family of languages} is an ordered pair $(\Sigma ,\mathcal{L} )$, such that
\begin{enumerate}
\item $\Sigma $ is an infinite set of symbols
\item every $L\in \mathcal{L} $ is a language over some finite set $\Sigma_1 \subset \Sigma $
\item $L \neq \es $ for some $L \in \mathcal{L} $
\end{enumerate}

\paragraph{}
\definicia A family of languages is called a \emph{(full) trio}, if it is closed under $\epsilon $-free (arbitrary) homomorphism, inverse homomorphism and intersection with a regular set.

\paragraph{}
\definicia A (full) trio is called a \emph{(full) semi-AFL}, if it is closed under union.

\paragraph{}
\definicia A (full) semi-AFL is called a \emph{(full) AFL}, if it is closed under concatenation and $+$.

%------------------------------------------------------------------------------------------
\section{Transformation models}
\paragraph{}
We shall now define some of the models mentioned in the Introduction. Although the central point of our interest is an a-transducer, we also introduce the definitions of other models, which will be used in the next chapters, because they can give us an insight of language transformations in general and many of the concepts used in results involving them can be put to use by examination of a-transducers.

\paragraph{}
Since all transformation models used in our thesis are, in fact, generalizations of an a-transducer, we define it first and then we only specify the differences between a-transducers and other models.

\paragraph{}
\definicia An \emph{a-transducer} is a 6-tuple $M=(K, \Sigma_{1}, \Sigma_{2}, H, q_{0}, F)$, where
\begin{itemize}
\item $K$ is a finite set of states,
\item $\Sigma_{1} $ and $\Sigma_{2} $ are the input and output alphabet, respectively,
\item $H \subseteq K \times \Sigma_{1}^{*} \times \Sigma_{2}^{*} \times K$ is the transition function, where $H$ is finite,
\item $q_{0} \in K$ is the initial state,
\item $F \subseteq K$ is a set of accepting states.
\end{itemize}
If $H \subseteq K \times \Sigma_{1}^{*} \times \Sigma_{2}^{+} \times K$, we call $M$ an \emph{$\epsilon $-free} a-transducer.

\paragraph{}
\definicia If $H \subseteq K \times \Sigma_{1} \cup \{ \epsilon \} \times \Sigma_{2} \cup \{ \epsilon \} \times K$, the corresponding a-transducer is called \emph{1-bounded}.

\paragraph{}
\definicia A \emph{configuration} of an a-transducer is a triple $(q, u, v)$, where $q \in K$ is a current internal state, $u \in \Sigma_{1}^{*}$ is the remaining part of the input and $v$ is the already written output.

\paragraph{}
\definicia A \emph{computational step} is a relation $\KV$ on configurations defined as follows: \\
\centerline{$(q, xu, v) \KV (p, u, vy) \Leftrightarrow (q, x, y, p) \in H$.}

\paragraph{}
\definicia An \emph{image} of a language $L$ by an a-transducer $M$ is a set \\
\centerline{$M(L) = \{ w|\exists u \in L, q_{F} \in F; (q_{0}, u, \epsilon) \V (q_{F}, \epsilon, w) \} $}

\paragraph{}
\definicia For $i=0,1,2,3, w \equiv (x_{0},x_{1},x_{2},x_{3}) \in H$, we define $pr_{i}(w) = x_{i}$ and call $pr_{i}$ the \emph{$i$-th projection} .

\paragraph{}
\definicia A \emph{computation} of an a-transducer $M$ is a word $h_{0}h_{1}...h_{m} \in H^{*}$, such that
\begin{enumerate}
\item $pr_{0}(h_{0}) = q_{0}$ ($q_{0}$ is the initial state of $M$),
\item $\forall i: pr_{3}(h_{i}) = pr_{0}(h_{i+1})$
\item $pr_{3}(h_{m}) \in F$
\end{enumerate}

\paragraph{}
\oznacenie We denote a language of all computations of $M$ by $\Pi_{M}$. Note, that $\Pi_{M}$ is regular (\cite{gin:AATPFL}).

\paragraph{}
\definicia Alternatively, we can define an image of $L$ by an a-transducer $M$ as \\
\centerline{$M(L) = \{ pr_{2}(pr_{1}^{-1}(w) \cap \Pi_{M} | w \in L \}$.}

\paragraph{}
\definicia An \emph{A-transduction} is a function $\Phi : \Sigma_{1}^{*} \rightarrow 2^{\Sigma_{2}^{*}}$ defined as follows: \\
\centerline{$\forall x \in \Sigma_{1}^{*}: \Phi(x) = \{ M(x) \} $.}

\paragraph{}
We have described the core model of our thesis, namely an a-transducer, and now we define two similar, but simpler models.

\paragraph{}
\definicia A \emph{sequential transducer} is a 7-tuple $M=(K, \Sigma_{1}, \Sigma_{2}, \delta, \sigma, q_{0}, F)$, where
\begin{itemize} 
\item $K, \Sigma_{1}, \Sigma_{2}, q_{0}, F$ are like in an a-transducer,
\item $\delta $ is a transition function, which maps $K \times \Sigma_{1} \rightarrow K$,
\item $\sigma $ is an output function, which maps $K \times \Sigma_{1} \rightarrow \Sigma_{2}^{*} $.
\end{itemize}

\paragraph{} 
A sequential transducer can be seen as a "deterministic" 1-bounded a-transducer, which set $H$ fulfills following conditions:
\begin{enumerate}
\item for every pair $(q, a) \in K \times \Sigma_{1}$, there is exactly one element $h \in H$, such that $pr_{0}(h) = q$ and $pr_{1}(h) = a$,
\item $\forall h \in H: pr_{1}(h) \neq \epsilon \wedge pr_{2}(h) \neq \epsilon $.
\end{enumerate}

\paragraph{}
\oznacenie By $\hat{\delta}$ and $\hat{\sigma}$ we denote an extension of $\delta $ ($\sigma $) on $K \times \Sigma_{1}^{*} $, defined recursively as \\
$\forall q \in K, w \in \Sigma_{1}^{*}, a \in \Sigma_{1}:$ \\
\begin{itemize} 
\item $\hat{\delta}(q, a) = \delta (q,a)$, $\hat{\delta}(q,wa) = \delta (\hat{\delta}(q, w), a)$,
\item $\hat{\sigma}(q, a) = \sigma (q,a)$, $\hat{\sigma}(q,wa) = \sigma (\hat{\delta}(q, w), a)$.
\end{itemize}

\paragraph{}
We omit the definitions of a configuration, computational step and image related to sequential transducers, since they are very similar to the a-transducer.

\paragraph{}
\definicia A \emph{sequential function} is a function represented by a sequential transducer. Formally, if $M=(K, \Sigma_{1}, \Sigma_{2}, \delta, \sigma, q_{0}, F)$ is a sequential transducer, then \\
\centerline{$\forall w \in \Sigma_{1}^{*}$, s. t. $\hat{\delta}(q_{0}, w) \in F$: $f_{M}(w) = \hat{\sigma}(q_{0}, w)$.}

\paragraph{}
We conclude this section with a definition of one more model, which can be viewed as a generalization of a sequential transducers.

\paragraph{}
\definicia A \emph{generalized sequential machine (gsm)} is a 6-tuple $M=(K, \Sigma_{1}, \Sigma_{2}, \delta, \sigma, q_{0})$, where $K, \Sigma_{1}, \Sigma_{2}, \delta, \sigma, q_{0}$ are as in sequential transducer.

\paragraph{}
As one can see, a generalized sequential machine is a sequential transducer with $F \equiv K$ and therefore all other concepts are defined just like in a sequential transducer.

\paragraph{}
\oznacenie A sequential function described by a generalized sequential machine is called a \emph{gsm mapping}.

%------------------------------------------------------------------------------------------
\section{Complexity, advisors and decomposition}

\paragraph{}
In this section we define the concept of advisors and decompositions.

\paragraph{}
\definicia The \emph{state complexity} of an a-transducer $M = (K, \Sigma_1, \Sigma_2, H, q_0, F)$ (a finite automaton $A = (K, \Sigma, \delta, q_0, F)$), denoted by $\C{T}$ ($\C{A}$), is the number of its states. Formally \\
\centerline{$\C{T} = |K|$.}

\paragraph{}
\definicia The \emph{state complexity} of a regular language $L$, denoted by $\C{L}$, is the state complexity of its minimal deterministic finite automaton. Formally \\
\centerline{$\C{L} = min\{ \C{A}|L(A) = L\} $.}

If $L$ is not regular, then $\C{L} = \infty $.

\paragraph{}
\definicia In a similar way, we can define the \emph{a-transducer state complexity} of a pair of languages $L_1, L_2$, denoted by $\C{L_1,L_2}$, as the state complexity of the minimal a-transducer, which translates language $L_1$ to $L_2$. Formally \\
\centerline{$\C{L_1, L_2} = min\{ \C{M}|M(L_1) = L_2\} $.} \\
Note, that it is possible, that $\C{L_1,L_2} \neq \C{L_2,L_1}$ and it may happen, that $\C{L_1,L_2} \allowbreak  = \infty$ (if there is no a-transducer $M$, such that $M(L_1) = L_2$).

\paragraph{}
Now we would like to define the acceptation of a language with an advisor and some other concepts presented in \cite{Gazi}.

\paragraph{}
\definicia For a language $L_1$ and an automaton $A = (K, \Sigma, \delta, q_0, F)$, a \emph{language accepted by $A$ with advisor $L_1$} is the language \\
\centerline{$L[L_1](A) =  \{ w \in L_1 | (q_0, w) \V_A (q, \epsilon), q \in F \}$.}\\
Another way for looking at this fact is, that $L[L_1](A) = L(A) \cap L_1$.
 
\paragraph{}
\definicia Let $A'=(K', \Sigma, \delta', q'_0, F')$ and $A=(K, \Sigma, \delta, q_0, F)$ be deterministic finite automata. We say, that $A'$ realizes the the state behavior of $A$, if there is an injective mapping $\alpha: K \rightarrow K'$, such that:

\begin{itemize}
\item $\forall a \in \Sigma, \forall q \in K: \delta(\alpha(q), a) = \alpha(\delta(q, a))$,
\item $\alpha(q_0) = q'_0$.
\end{itemize}

Moreover, if $\forall q \in K: \alpha(q)\in F' \Leftrightarrow q \in F$, we say, that $A'$ \emph{realizes the state and acceptation behavior} of $A$.

\paragraph{}
\definicia Let $A_1=(K_1, \Sigma, \delta_1, q_1, F_1), A_2=(K_2, \Sigma, \delta_2, q_2, F_2)$ be deterministic finite automata. Their \emph{parallel connection} is an automaton $A_1 || A_2 = (K_1\times K_2, \Sigma, \delta, (q_1,q_2), F_1\times F_2)$, where $\forall (p_1,p_2) \in K_1\times K_2, a \in \Sigma: \delta((q_1,q_2), a) = (\delta_1(p_1, a), \delta_2(p_2,a))$.

\paragraph{}
\definicia We say, that a pair $(A_1, A_2)$ is a \emph{state behavior (SB-) decomposition} of a deterministic finite automaton $A$, if $A_1 || A_2$ realizes the state behavior of $A$. If $A_1 || A_2$ realizes the state and acceptance behvaior of $A$, $(A_1,A_2)$ forms a \emph{state and acceptance (ASB-) behavior decomposition}.\\
If $\C{A_1} < \C{A}$ and $\C{A_2} < \C{A}$, the decomposition is called \emph{nontrivial}.

\paragraph{}
\definicia A language $L$ and its corresponding minimal deterministic finite automaton $A$ are called (A)SB-undecomposable, if $A$ has no nontrivial (A)SB-decomposition. The class of all regular (A)SB-undecomposable languages is denoted by $\mathcal{U}_{SB}$ ($\mathcal{U}_{ASB}$).

\section{Group and number theory}
\paragraph{}
\color{red}ToDo: mozno definovat aspon zakladne pojmy z teorie grup, kedze ich neskor pouzivam a aj vysledok gcd(a,b).lcm(a,b)=a.b\color{black}

\paragraph{}
