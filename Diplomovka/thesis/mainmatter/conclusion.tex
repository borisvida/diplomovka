\chapter*{Conclusion}
\addcontentsline{toc}{chapter}{Conclusion}
\label{chap:conclusion}

\paragraph{}
In our thesis we studied the use of transformation in solving problems with supplementary information. We have presented three formalizations of this idea using deterministic finite automata and sequential and a-transducers. We have briefly examined these three frameworks and compared some of the correspoding classes of decomposable and undecomposable languages. We have moreover compared these classes to previously know class of A-decomposable languages and to the class of regular languages $\R$. Furthermore, we have presented an original result concerning the complexity of a-transducers. In the last Section we have examined few closure properties of T-decomposable languages under basic operations.

\paragraph{}
There are many possibilities of further research in this area. One of them is to examine further properties of presented classes of languages. Another one is to find a necessary and/or sufficient conditions on $T$-, $NT_{\forall}$- nad $NT_{\exists}$-decomposability of regular languages. Moreover, probably an interesting direction of research would be looking for classes of languages, that can be decomposed with similar advice, or with the use of a fixed type of transformation (change of the alphabet etc.).
