\documentclass[12pt,oneside,a4paper]{book}
\usepackage{mathtools}
\usepackage[top=2.5cm,bottom=2.5cm,right=2cm,left=3.5cm]{geometry}
\usepackage[utf8]{inputenc}
\usepackage{graphicx}
\usepackage{pdfpages}
%\usepackage{lmodern}
\usepackage{txfonts} %Times New Roman
\usepackage[T1]{fontenc}
\usepackage[english,slovak]{babel}
\usepackage{amsmath}
\usepackage{amssymb}
\usepackage{mathrsfs}
\linespread{1.3} %riadkovanie 1.5
\input foja
\begin{document}

\paragraph{}
\definicia A sequence of generalized sequential machines $\mathcal{G} = \{G_n:n \in \mathbb{N}\}$ is called \emph{logspace uniform}, if there is a deterministic logspace-bounded Turing machine, which for all $n$, outputs the description of $G_{n}$ on the input $1^n$.

\paragraph{}
\definicia A \emph{translator} is a logspace(?) uniform sequence of generalized sequential machines.

\paragraph{}
Description of the framework: Let us have a sequence of gsm $\mathcal{G} = \{G_n:n \in \mathbb{N}\}$ and a deterministic finite automaton(?) $V$ (verifier), which has to decide, wheter an input $w$ belongs to a language $L_{dec}$. There is an advisor $A$, which has some information of the input $w$ (the information is in the form of a language $L_{inf}$). $L_{inf}$ is sent to $V$ together with an index $i$ to the sequence $\mathcal{G}$. $V$ then transforms $L_{inf}$ using a gsm $G_{i}$ and gets the language $L_{adv}$. Now, $V$ knows, that $w$ belongs to $L_{adv}$ and this can possibly help to reduce its complexity.

\paragraph{}
\definicia The \emph{state complexity} of a gsm $G = (Q, \Sigma, \Delta, \delta, \lambda, q_1)$, denoted by $\mathscr{C}_{state}(G)$, is the number of its states. Formally \\
\centerline{$\mathscr{C}_{state}(G) = |Q|$}

\paragraph{}
\oznacenie By $L = L[L_{adv}](V)$ we denote the fact, that $V$ decides the language $L$ with the advisory information, that the input belongs to $L_{adv}$.

\paragraph{}
\cpriklad Let $L = (a^6)^*$ and $V = (Q, \Sigma, \delta, q_0, F)$, where $Q = {q_0, q_1, q_2}$, $\Sigma = {a}$, $F = {q_0}$ and $\delta (q_i, a) = q_{i+1 mod 3}$ for $i = 0,1,2$. Moreover, let $L_{adv} = (a^2)^*$. Then, althought $L \neq L(V)$, it is easy to see, that $L = L[L_{adv}](V)$.

\paragraph{}
\definicia For a fixed sequence $\mathcal{G}$, a language $L_{inf}$ with an index $i$ is an \emph{effective advice with regard to $L_{dec}$}, if there exists a verifier $V$ with $k$ states, such that $L = L[L_{adv}](V)$, the minimal DFA for $L_{dec}$ has $l$ states and $\mathscr{C}_{state}(G_i) < l - k$.

\paragraph{}
\cpriklad Let $\mathcal{G}$ be a sequence of gsm, where $G_{i}$ is a gsm, that computes a bitwise XOR of the input and a key $k$, which is obtained as follows: take a binary representation of $i$ and remove the leading 1 (otherwise it would not be possible to have keys with initial sequence of zeros). If the input is longer than the key, we compute the bitwise XOR with $k^*$.

\paragraph{}
\clema If a key $k$ of length $n$ is nonperiodical (it has not a form $k'^i$ for $i > 1$), then a gsm from example 1 with key $k$ has at least $n$ states.

\paragraph{}
\dokaz Let $G = (Q, \Sigma, \Delta, \delta, \lambda, q_1)$ be a gsm, where $Q = \{q_1, ..., q_k\}, \Sigma = \Delta = \{0, 1\}, \delta(q_i, a) = q_{(i mod k)+1}$ and $\lambda(q_i, a) = k_i \oplus a$ for all $a \in \Sigma$ and $1 \leq i \leq n$ (where $k_i$  is the $i$-th bit of the key $k$). It is easy to see, that $G$ computes the correct output.

\paragraph{}
Now, we would like to show, that $G$ is the  minimal gsm for this sequential function. Let $G'$ be a gsm with $n-1$ states, such that $G'(w) = G(w)$ for all $w \in \{0,1\}^*$. Let us take a look at a computation of $G'$ on an input $0^n$ - so the output should be $k$. $G'$ clearly uses at least one of its states to output two (or more) letters (not necessarily consecutive). 

\color{red}TODO: finish\color{black}\\
\square

\paragraph{}
\clema For each $n$ and $l$, which is a divisor of $n$, there exists a gsm $G_i$ from Example 1, such that $G_i$ uses a key $k, |k| = n$ and $\mathscr{C}_{state}(G_i) = l$.

\paragraph{}
\dokaz

\color{red}TODO: periodic keys\color{black}\\
\square

\paragraph{}
\cdosledok Let $f_{\mathcal{G}}$ be a function defined as follows: $f_{\mathcal{G}}(i) = \mathscr{C}_{state}(G_i)$. Then, $f_{\mathcal{G_i}}$ is not monotonic.

\paragraph{}
\cpriklad Let $\mathcal{H}$ be a sequence of gsm, where $H_{k}$ transforms the input from a $k$-ary alphabet $a_1, a_2, ..., a_k$ to a binary alphabet $0,1$, using a standard encoding (i. e. $a_i$ is encoded as binary encoded number $i$ with added leading zeros, such that the length of the code is $\lceil log_2 k \rceil$).

\paragraph{}
Is is easy to see, that such gsm needs only one state, which transforms letters $a_k$ to its binary code.

\paragraph{}
For this reason, it is sometimes convenient to measure the complexity not only by state count, but rather by complexity of the output function. We will call this measure the output complexity.

\paragraph{}
\definicia The \emph{output complexity} of a gsm $G = (Q, \Sigma, \Delta, \delta, \lambda, q_1)$, denoted by $\mathscr{C}_{out}(G)$, is the sum of output length of its transitions. More formally \\
\centerline{$\mathscr{C}_{out}(G) = \sum_{q\times a \rightarrow w \in \lambda} (1 + |w|)$}

\paragraph{}
\clema For $1 \leq i \leq k$ and $H_i$ from Example 2, $\mathscr{C}_{out}(H_i) = i*\lceil \log i \rceil$.

\paragraph{}
\dokaz

\color{red}TODO: add proof\color{black}\\
\square.

\paragraph{}
\cdosledok Let $f_{\mathcal{H}}$ be a function defined as follows: $f_{\mathcal{H}}(i) = \mathscr{C}_{out}(H_i)$. Then, $f_{\mathcal{H}}$ is monotonic.

\paragraph{}

\end{document}
