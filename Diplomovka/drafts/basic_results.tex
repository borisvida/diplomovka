\documentclass[12pt,oneside,a4paper]{book}
\usepackage{mathtools}
\usepackage[top=2.5cm,bottom=2.5cm,right=2cm,left=3.5cm]{geometry}
\usepackage[utf8]{inputenc}
\usepackage{graphicx}
\usepackage{pdfpages}
%\usepackage{lmodern}
\usepackage{txfonts} %Times New Roman
\usepackage[T1]{fontenc}
\usepackage[english,slovak]{babel}
\usepackage{amsmath}
\usepackage{amssymb}
\usepackage{mathrsfs}
\linespread{1.3} %riadkovanie 1.5
\input foja
\begin{document}

\paragraph{}
In this part of our thesis, we would like to compare our setting to the setting presented by [Gazi, diploma thesis] (see Section [number of definitions Section]).

\paragraph{}
\cdefinicia The language $L$ is called \emph{T-decomposable}, if there is a language $L_{adv}$, which is an effective advice for $L$.\\
Otherwise, we call $L$ \emph{T-undecomposable}.

%\paragraph{}
%\cdefinicia The language $L$ is called \emph{T-undecomposable}, if there is no language $L_{adv}$, which would be an effective advice for $L$. Formally \\
%\centerline{$min\{ \mathscr{C}_{state}(A)|L[T^{-1}(L_adv)](A) = L; T,L_{adv}$ arbitrary$\} = \mathscr{C}_{state}(L)$.}

\paragraph{}
\cveta If there exists a non-trivial $ASB$-decomposition for language $L$ \color{red}[would like to write it in different way (using advisors, not decomposition), but did not find suitable definitions in Gazi's diploma thesis]\color{black}, then $L$ is T-decomposable.

\paragraph{}
\dokaz Easy to see, using an A-transducer computing the identity. \square

\paragraph{}
However, the next theorem shows, that the reverse implication does not hold.

\paragraph{}
\cveta There are infinitely many T-decomposable languages, thate are not $ASB$-decomposable.

\paragraph{}
\dokaz Such languages are for example $L_{x} = \{ u\$ xv | u,v \in \{ a,b\}^* \}$ for a fixed string $x \in \{ a,b\}^*, |x| \geq 14$ and even.

\paragraph{}
We prove this claim in two steps. First, we need to show, that $L_{x}$ is T-decomposable. It is easy to see, that  a DFA accepting $L_{x}$ needs at least $|x| + 1$ states, therefore $\mathscr{C}_{state}(L_x) = |x|+1$.

\paragraph{}
However, we can use an advice to simplify the accepting automaton as follows: our A-transducer $T$ will read the input word in the initial state with no output, until it finds the special marker $\$ $. Then, using another three states, it encodes pairs of symbols (i. e. sequences $aa, ab, ba, bb$) into new letters $c, d, e, f$, respectively. If there is just one symbol in the end, $T$ will read it and traverse into accepting state $q_F$ with no further transitions, otherwise it will make an $\epsilon $-transition into $q_{F}$. Note, that $T$ uses just five states.

\paragraph{}
Now, the advise language $L_{x,adv} = \{ x'v | x'$ is the aforementioned encoded form of $x$ into symbols $c,d,e,f \}$. Clearly, $|x'| = \frac{|x|}{2}$ and $\mathscr{C}_{state}(L_{x,adv}) = \frac{|x|}{2}+1$.

\paragraph{}
The decider $D$ needs construct just an automaton for $\{a,b,\$\}^*$, since the advice gives full information about $L_x$. Alltogether, we used $5 + \frac{|x|}{2}+1+1$ states, therefore for $|x| \geq 14$ is $L_{x,adv}$ with $T$ an effective advice with regard to $L_x$.

\paragraph{}
Our next goal is to show, that $L_x$ is not $ASB$-decomposable. \color{red}[ToDo: directly or using necessary conditions from Gazi]\color{black}
\square

\paragraph{}
\cdosledok There are infinitely many T-decomposable languages.

\paragraph{}
\cveta There are infinitely many T-undecomposable languages.

\paragraph{}
To prove this theorem, we show the following lemma.

\paragraph{}
\clema Each of the languages $L_{n} = \{ w \in \{ a, b\}^* | \#_{a}(w) \mod p \equiv 0, p$ is the $n$-th prime$\}$ is T-undecomposable.

\paragraph{}
\dokaz

\color{red}TODO: proof - probably using some alternation of pumping lemma, try to show, that the number of equivalence classes can't be reduced, since  $p$ is prime\color{black}\\
\square

\paragraph{}
As we have seen, the class of regular languages can be divided into three parts - both $ASB$-decomposable and T-decomposable languages; $ASB$-undecomposable, but T-decomposable; and T-undecomposable languages \color{red}[I will try to write this in a better way]\color{black}. In the next part of our thesis, we would like to investigate some properties of these classes.

\end{document}
